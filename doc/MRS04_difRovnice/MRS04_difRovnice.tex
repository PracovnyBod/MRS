\documentclass[a4paper, 10pt, ]{article}

\input{../misc_LaTeX/preamble.tex}

\def\oznacenieCasti{MRS04 - ZS2024}


\usepackage{longtable}




\begin{document}


\lstset{%
style=mystyle,
rangebeginprefix=\#\#\#\ cellB\ ,%
rangebeginsuffix=\ \#\#\#,%
rangeendprefix=\#\#\#\ cellE\ ,%
rangeendsuffix=\ \#\#\#,%
includerangemarker=false,
}





\fontsize{12pt}{22pt}\selectfont

\centerline{\textsf{Modelovanie a riadenie systémov} \hfill \textsf{\oznacenieCasti}}

\fontsize{18pt}{22pt}\selectfont





\begin{flushleft}
	\textbf{\textsf{Diferenciálne rovnice}}
\end{flushleft}





\normalsize

\bigskip

{\hypersetup{hidelinks}

\tableofcontents

}

\bigskip

\vspace{18pt}



\noindent
\lettrine[lines=3, nindent=0pt]{C}{ieľom} tohto textu je v rámci predmetu stanoviť pojem diferenciálna rovnica. Diferenciálna rovnica bude slúžiť ako matematický opis dynamického systému. V prvom rade je potrebné zaoberať riešením diferenciálnej rovnice. Je potrebné študovať čo je to riešenie diferenciálnej rovnice a následne spôsoby a metódy ako riešenie diferenciálnej rovnice nájsť.





\section{Úvodné pojmy}

Majme rovnicu
\begin{equation} \label{eq:1}
    \frac{\text{d}y(t)}{\text{d}t} + a y(t) = 0
\end{equation}
Neznámou v tejto rovnici je funkcia času $y(t)$. Symbol $t$ označuje čas. Koeficient $a \in \mathbb R$ je reálne číslo (nie je to funkcia času). 

Prvý člen ľavej strany rovnici je časová derivácia funkcie $y(t)$. 

Diferenciálnou túto rovnicu nazývame preto, že v rovnici sa vyskytuje derivácia neznámej funkcie $y(t)$.


Deriváciu funkcie $y(t)$ podľa času označujeme $\frac{\text{d}y(t)}{\text{d}t}$ alebo $\dot y(t)$. Druhá derivácia by bola označená $\frac{\text{d}^2y(t)}{\text{d}t^2}$ alebo $\ddot y(t)$ a tak ďalej. Rovnica \eqref{eq:1} by teda mohla byť zapísaná aj v tvare
\begin{equation}
    \dot y(t) + a y(t) = 0
\end{equation}


Navyše, rovnica \eqref{eq:1} je:
\begin{itemize}[leftmargin=0pt, labelsep=3mm, itemsep=0pt]
    \item Obyčajná diferenciálna rovnica (ordinary differential equation, ODE). \newline Obyčajnou je dif. rovnica vtedy, keď neznáma funkcia je funkciou len jednej premennej (v tomto prípade len času $t$).
    \item  Rovnica prvého rádu. \newline Rovnica je prvého rádu, pretože najvyššia derivácia neznámej funkcie je prvého rádu (order of derivative). Rád rovnice je určený najvyššou deriváciou neznámej.
    \item Lineárna rovnica. \newline Je možné povedať, že neznáma funkcia $y(t)$ a jej derivácie sa vyskytujú len v lineárnych kombináciách. Hovoríme, že rovnica je lineárna. Užitočným je aj pohľad keď rovnicu upravíme vo všeobecnosti na tvar kde na ľavej strane je najvyššia derivácia neznámej, v~tomto prípade prvá derivácia a pravú stranu považujeme za funkciu zvyšných derivácií neznámej, v tomto prípade nultej derivácie.
    \begin{equation}
        \frac{\text{d}y(t)}{\text{d}t} = f\left(y(t) \right)
    \end{equation}
    ak táto funkcia $f$ je lineárna, potom je rovnica lineárna.
\end{itemize}



\paragraph{Homogénna a nehomogénna rovnica}

Rovnica \eqref{eq:1} je homogénna. Homogénnou je rovnica vtedy, keď sa v rovnici nachádza len funkcia času, ktorá je neznámou. Iné funkcie času sa v rovnici nevyskytujú. Z hľadiska systému to znamená, že systém má len výstup, len výstupný signál $y(t)$. Schematicky znázornené:
\begin{center}

    \vbox{%
    \makebox[\textwidth][c]{%
	\input{../fig_standalone/schB_sys_outonly.pdf_tex}
	}

	\figcaption{Systém s výstupným signálom (s výstupnou veličinou)}
    \label{schB_sys_outonly}
    }

\end{center}

Príkladom nehomogénnej rovnice by mohla byť rovnica
\begin{equation} \label{eq:2}
    \dot y(t) + a y(t) = u(t) 
\end{equation}
kde $u(t)$  je funkcia času. Nie je to však neznáma funkcia času. Jedinou neznámou v~rovnici ostáva $y(t)$. Rovnica \eqref{eq:2} je nehomogénna, pretože obsahuje aj iné funkcie času ako len neznámu.

Z hľadiska systému to znamená, že systém má aj vstup, vstupný signál $u(t)$. Schematicky znázornené:

\begin{center}

    \vbox{%
    \makebox[\textwidth][c]{%
	\input{../fig_standalone/schB_sys_SISO.pdf_tex}
	}

	\figcaption{Systém s jedným výstupným signálom a jedným výstupným signálom.}
    \label{schB_sys_SISO}
    }

\end{center}


\paragraph{Začiatočná podmienka}

Časová funkcia $y(t)$ nadobúda hodnotu v akomkoľvek čase $t$. Konkrétnej hodnote takpovediac „na začiatku času“, formálne v čase $t=0$, hovoríme začiatočná podmienka. Formálne
\begin{equation}
    y(0) = y_0  
\end{equation}
kde $y_0$ je konkrétna hodnota funkcie $y(t)$ v čase $t=0$.


\paragraph{Rovnice vyššieho rádu}

Príkladom dif. rovnice druhého rádu, homogénnej, by mohla byť
\begin{equation}
    \ddot y(t) + a_1 \dot y(t) + a_0 y(t) = 0
\end{equation}
kde $a_1$, $a_0$ sú koeficienty, reálne čísla. 

Príkladom nehomogénnej dif. rovnice druhého rádu by mohla byť
\begin{equation}
    \ddot y(t) + a_1 \dot y(t) + a_0 y(t) = u(t)
\end{equation}
kde $u(t)$  je funkcia času predstavujúca vstupný signál systému. Vo všeobecnosti je možné uvažovať aj derivácie vstupného signálu $u(t)$, teda napríklad
\begin{equation}
    \ddot y(t) + a_1 \dot y(t) + a_0 y(t) = u(t) + \dot u(t)
\end{equation}
čo však stále je dif. rovnica druhého rádu, samozrejme nehomogénna.

Úplne vo všeobecnosti, obyčajná lineárna nehomogénna diferenciálna rovnica $n$-tého rádu:
\begin{equation}
    a_n \frac{\text{d}^n y(t)}{\text{d}t^n} 
    + a_{n-1} \frac{\text{d}^{n-1} y(t)}{\text{d}t^{n-1}}
    +
    \cdots
    +
    a_0 y(t)
    =
    b_m \frac{\text{d}^m u(t)}{\text{d}t^m} 
    + b_{m-1} \frac{\text{d}^{m-1} u(t)}{\text{d}t^{m-1}}
    +
    \cdots
    +
    b_0 u(t)
\end{equation}
Koeficienty $a_n, \ldots, a_0$ a $b_m, \ldots, b_0$ sú konštantné. Číslo $n$ udáva najvyššiu časovú deriváciu výstupného signálu a číslo $m$ udáva najvyššiu časovú deriváciu vstupného signálu. Uvažujeme, že rovnica opisuje \emph{kauzálny} dynamický systém preto musí platiť $n>m$.























\section[Schematické znázornenie diferenciálnej rovnice]{Schematické znázornenie diferenciálnej rovnice\\(dynamického systému)}

\subsection{Homogénna diferenciálna rovnica prvého rádu}


Majme diferenciálnu rovnicu v tvare:
\begin{equation} \label{hodrpr01}
    \frac{\text{d}y(t)}{\text{d}t} = - a y(t) \qquad y(0)=y_0
\end{equation}
kde $y(t)$ je neznáma funkcia času, riešením rovnice hľadáme túto funkciu času, a $a\in\mathbb{R}$ (je to reálne číslo) je daný parameter. Hodnota $y_0$ v začiatočnej podmienke je tiež daná.

Ide o homogénnu diferenciálnu rovnicu pretože neobsahuje žiadne iné členy len také, ktoré obsahujú neznámu $y(t)$. Ak by sme členy obsahujúce len neznámu presunuli na ľavú stranu, teda:
\begin{equation}
    \frac{\text{d}y(t)}{\text{d}t} + a y(t) = 0
\end{equation}
na pravej strane je nula. Rovnica je homogénna.

Z hľadiska systémového je $y(t)$ výstupnou veličinou (výstupným signálom) a číslo $a$ je parameter systému. $y(0)=y_0$ je začiatočná podmienka kde $y_0$ je hodnota signálu $y(t)$ v čase $t=0$.

V tomto prípade nemáme vstupnú veličinu systému (vstupný signál). Ako sme uviedli, z pohľadu diferenciálnej rovnice, alebo z hľadiska pojmu dynamický systém, hovoríme o homogénnej diferenciálnej rovnici. Obsahuje len samotnú neznámu (výstupnú veličinu, výstupný signál).

Dôležitým faktom je, že táto rovnica sa nazýva diferenciálna rovnica prvého rádu, pretože najvyššia derivácia neznámej je prvého rádu.

Diferenciálnu rovnicu, alebo z iného hľadiska dynamický systém v tvare \eqref{hodrpr01}, je možné schematicky znázorniť s využitím základných funkčných blokov, stavebných prvkov, medzi ktorými sú signály (v čase premenlivé veličiny).

Výsledná schéma je nasledovná:

\begin{center}

    \vbox{%
        \makebox[\textwidth][c]{%
            \input{../fig_standalone/RCclanok_v2_sch_v2.pdf_tex}
        }
        \figcaption{Schéma dynamického systému zodpovedajúca rovnici \eqref{hodrpr01}.}
    }
	\label{RCclanok_v2_sch_v2}

\end{center}


\subsection{Prvky blokovej schémy}

Dynamický systém, alebo diferenciálnu rovnicu, ktorá ho opisuje, je možné znázorniť aj graficky \emph{blokovou schémou}. Takou, ktorá prípadne umožňuje aj opačný postup, teda schéma určuje diferenciálnu rovnicu. Výsledná bloková schéma taktiež umožňuje ďalšiu analýzu dynamického systému.

Typicky sa v rámci takejto schémy používajú nasledovné prvky a bloky.

 \paragraph{Signál}

 \begin{center}

    \vbox{%
    \makebox[\textwidth][c]{%
	\input{../fig_standalone/schB_signal.pdf_tex}
	}

	\figcaption{Signál v blokovej schéme.}
    \label{schB_signal.pdf}
    }

\end{center}

\noindent
Signál je reprezentovaný čiarou so šípkou, ktorá určuje smer prenosu informácie. Pri čiare je uvedené označenie signálu.


\paragraph{Zosilňovač}

\begin{center}

    \vbox{%
    \makebox[\textwidth][c]{%
	\input{../fig_standalone/schB_gain.pdf_tex}
	}

	\figcaption{Zosilňovač v blokovej schéme.}
    \label{schB_gain.pdf}
    }

\end{center}

\noindent
Ide o blok, ktorý vo všeobecnosti zosilňuje signál na vstupe tohto bloku. Samozrejme, môže ísť aj o „zoslabenie“. Inými slovami, tento blok vynásobí hodnotu vstupného signálu $u(t)$ hodnotou parametra $a$ a výstupom je signál $y(t)$. Matematicky zapísané
\begin{equation}
    y(t) = a \,  u(t)
\end{equation}
Parameter $a$ môže mať ľubovolnú hodnotu, môže byť menší ako $1$ (zoslabenie) alebo záporný.



\paragraph{Sumátor}

\begin{center}

    \vbox{%
    \makebox[\textwidth][c]{%
	\input{../fig_standalone/schB_sumator.pdf_tex}
	}

	\figcaption{Sumátor v blokovej schéme.}
    \label{schB_sumator.pdf}
    }

\end{center}

\noindent
Realizuje sčítanie hodnôt dvoch alebo viacerých signálov. Prípadne odčítanie. Príklad matematického zápisu:
\begin{equation}
    y(t) = u_1(t) + u_2(t)
\end{equation}



\paragraph{Integrátor}

\begin{center}

    \vbox{%
    \makebox[\textwidth][c]{%
    \input{../fig_standalone/schB_integrator.pdf_tex}
    }

    \figcaption{Integrátor v blokovej schéme.}
    \label{schB_integrator.pdf}
    }

\end{center}

\noindent
Predstavuje časovú integráciu vstupného signálu v zmysle matematického zápisu
\begin{equation}
    y(t) = \int u(t) \, \text{d}t \qquad y(0) = y_0
\end{equation}
kde $y_0$ je začiatočná hodnota signálu $y(t)$, teda začiatočná podmienka.


\paragraph{Derivácia}

Pre úplnosť uvedieme aj blok realizujúci časovú deriváciu signálu. Je opakom integrátora. Tento blok sa však v praxi používa menej často. Dôvodom je, že implementovať časovú deriváciu je v praxi

\begin{center}

    \vbox{%
    \makebox[\textwidth][c]{%
    \input{../fig_standalone/schB_derivacia.pdf_tex}
    }

    \figcaption{Časová derivácia v blokovej schéme.}
    \label{schB_derivacia.pdf}
    }

\end{center}






\subsection{Homogénna diferenciálna rovnica prvého rádu: príklad postupu zostavenia blokovej schémy}


Uvažujme dynamický systém daný diferenciálnou rovnicou v tvare
\begin{equation} \label{eq:HDR1R}
    \dot y(t) + a y(t) = 0 \qquad y(0) = y_0
\end{equation}
Postup pre zostavenie blokovej schémy dynamického systému môže byť nasledovný.

Rovnica \eqref{eq:HDR1R} je prvého rádu a neznámou je časová funkcia $y(t)$. Prepíšme rovnicu \eqref{eq:HDR1R} tak, aby na ľavej strane bola len najvyššia derivácia neznámej, v tomto prípade signál $\dot y(t)$. Teda
\begin{equation} \label{eq:HDR1Rs}
    \dot y(t) = - a y(t)
\end{equation}
Rovnica v tomto tvare je východiskom pre zostavenie blokovej schémy. Je totiž zrejmé, že signál $\dot y(t)$  existuje. Inými slovami, to, čo určite máme k dispozícii je signál $\dot y(t)$. Ak by tento signál neexistoval, tak vlastne rovnica \eqref{eq:HDR1Rs} by bola nezmyslom. V schéme teda máme k dispozícii signál $\dot y(t)$.
\begin{center}

    \vbox{%
    \makebox[\textwidth][c]{%
    \input{../fig_standalone/schB_pr0_k1.pdf_tex}
    }

    \vspace{-5mm}

    \figcaption{Bloková schéma rovnice \eqref{eq:HDR1R}, krok prvý.}
    \label{schB_pr0_k1.pdf}
    }

\end{center}
Rovnica v tvare \eqref{eq:HDR1Rs} tiež priamo ukazuje, že signál $\dot y(t)$ je to isté ako výraz $- a y(t)$. Vieme zostaviť blokovú schému tohto výrazu? Ide zjavne o zosilňovač so zosilnením $-a$, ktorý má na vstupe signál $y(t)$. Pridajme do blokovej schémy:
\begin{center}

    \vbox{%
    \makebox[\textwidth][c]{%
    \input{../fig_standalone/schB_pr0_k2.pdf_tex}
    }

    \figcaption{Bloková schéma rovnice \eqref{eq:HDR1R}, krok druhý.}
    \label{schB_pr0_k2.pdf}
    }

\end{center}
Keďže doslova $\dot y(t) = - a y(t)$, tak
\begin{center}

    \vbox{%
    \makebox[\textwidth][c]{%
    \input{../fig_standalone/schB_pr0_k3.pdf_tex}
    }

    \figcaption{Bloková schéma rovnice \eqref{eq:HDR1R}, krok tretí.}
    \label{schB_pr0_k3.pdf}
    }

\end{center}
Pripomeňme, že signál $\dot y(t)$ takpovediac existuje, je k dispozícii. Samotný signál $y(t)$ však nie je k dispozícii. Je potrebné ho vytvoriť z toho, čo už k dispozícii je. Je zrejmé, že signál $y(t)$ je možné získať integrovaním $\dot y(t)$, teda
\begin{center}

    \vbox{%
    \makebox[\textwidth][c]{%
    \input{../fig_standalone/schB_pr0_k4.pdf_tex}
    }

    \figcaption{Bloková schéma rovnice \eqref{eq:HDR1R}, krok štvrtý.}
    \label{schB_pr0_k4.pdf}
    }

\end{center}
pričom integrátor musí mať začiatočnú podmienku $y(0) = y_0$ (podľa \eqref{eq:HDR1R}).
\begin{center}

    \vbox{%
    \makebox[\textwidth][c]{%
    \input{../fig_standalone/schB_pr0_k4b.pdf_tex}
    }

    \figcaption{}
    \label{schB_pr0_k4b.pdf}
    }

\end{center}
Napokon
\begin{center}

    \vbox{%
    \makebox[\textwidth][c]{%
    \input{../fig_standalone/schB_pr0_k5.pdf_tex}
    }

    \figcaption{Bloková schéma rovnice \eqref{eq:HDR1R}.}
    \label{schB_pr0_k5.pdf}
    }

\end{center}
je bloková schéma dynamického systému, ktorá zodpovedá diferenciálnej rovnici \eqref{eq:HDR1R}.






\subsection{Nemogénna diferenciálna rovnica prvého rádu: príklad postupu zostavenia blokovej schémy}


Uvažujme dynamický systém daný diferenciálnou rovnicou v tvare
\begin{equation} \label{eq:DR1R}
    \dot y(t) + a y(t) = b u(t) \qquad y(0) = y_0
\end{equation}
kde $a$, $b$ sú konštanty a $u(t)$ je známy vstupný signál.

Rovnicu \eqref{eq:DR1R} prepíšme tak, aby na ľavej strane bola len najvyššia derivácia neznámej, teda signál $\dot y(t)$. Teda
\begin{equation} \label{eq:DR1Rs}
    \dot y(t) = - a y(t) + b u(t)
\end{equation}
Na začiatku máme k dispozícii signál $\dot y(t)$, teda
\begin{center}

    \vbox{%
    \makebox[\textwidth][c]{%
    \input{../fig_standalone/schB_pr1_k1.pdf_tex}
    }

    \vspace{-15mm}

    \figcaption{Bloková schéma rovnice \eqref{eq:DR1R}, krok prvý.}
    \label{schB_pr1_k1.pdf}
    }

\end{center}
Signál $\dot y(t)$ je súčtom dvoch iných signálov.
\begin{center}

    \vbox{%
    \makebox[\textwidth][c]{%
    \input{../fig_standalone/schB_pr1_k2.pdf_tex}
    }

    % \vspace{-5mm}

    \figcaption{Bloková schéma rovnice \eqref{eq:DR1R}, krok druhý.}
    \label{schB_pr1_k2.pdf}
    }

\end{center}
Prvý signál získame zosilnením signálu $y(t)$ zosilňovačom s~parametrom $-a$.
\begin{center}

    \vbox{%
    \makebox[\textwidth][c]{%
    \input{../fig_standalone/schB_pr1_k3.pdf_tex}
    }

    \figcaption{Bloková schéma rovnice \eqref{eq:DR1R}, krok tretí.}
    \label{schB_pr1_k3.pdf}
    }

\end{center}
Druhý získame zosilnením známeho (dostupného) signálu $u(t)$ zosilňovačom s~parametrom $b$.
\begin{center}

    \vbox{%
    \makebox[\textwidth][c]{%
    \input{../fig_standalone/schB_pr1_k4.pdf_tex}
    }

    \figcaption{Bloková schéma rovnice \eqref{eq:DR1R}, krok štvrtý.}
    \label{schB_pr1_k4.pdf}
    }

\end{center}
Signál $y(t)$ v tomto kroku však nie je dostupný, je potrebné ho vytvoriť z toho, čo už k dispozícii je. Signál $y(t)$ je možné získať integrovaním signálu $\dot y(t)$.
\begin{center}

    \vbox{%
    \makebox[\textwidth][c]{%
    \input{../fig_standalone/schB_pr1_kf.pdf_tex}
    }

    \figcaption{Bloková schéma rovnice \eqref{eq:DR1R}.}
    \label{schB_pr1_kf.pdf}
    }

\end{center}
Integrátor musí mať začiatočnú podmienku $y(0) = y_0$ (podľa \eqref{eq:DR1R}).




















\section{Rozklad na sústavu diferenciálnych rovníc prvého rádu}



Vo všeobecnosti platí, že každú diferenciálnu rovnicu vyššieho rádu je možné rozložiť (prepísať, transformovať) na sústavu rovníc prvého rádu. Ich počet je minimálne $n$.

Ako príklad uvažujme diferenciálnu rovnicu v tvare
\begin{equation} \label{povonadr2}
    a_2 \ddot y(t) + a_1 \dot y(t) + a_0 y(t) = b_0 u(t)
\end{equation}

Úlohou je rozložiť túto diferenciálnu rovnicu druhého rádu na dve diferenciálne rovnice prvého rádu.

Pre tieto dve nové rovnice je potrebné uvažovať dve veličiny, ktoré budú na mieste neznámej v nových diferenciálnych rovniciach. Označme ich $x_1(t)$ a $x_2(t)$.

Hľadáme teda dve diferenciálne rovnice, inak vyjadrené hľadáme
\begin{align*}
    \dot x_1(t) &= \quad ? \\
    \dot x_2(t) &= \quad ? 
\end{align*}
pritom na pravej strane je potrebné mať len také členy, ktoré obsahujú len nové veličiny $x_1(t)$ a $x_2(t)$ a neobsahujú pôvodnú veličinu $y(t)$. Mimochodom, veličina $u(t)$, teda vstupný signál systému, je z hľadiska riešenia diferenciálnej rovnice známa. Neznámou je výstupná veličina $y(t)$. Signál $u(t)$ preto môže nezmenený figurovať v nových hľadaných diferenciálnych rovniciach. Dosiahnuť uvedené je možné nasledujúcim postupom. 

Ako prvé \emph{zvoľme}
\begin{equation} \label{volba1}
    x_1(t) = y(t)
\end{equation}
To znamená
\begin{equation}
    \dot x_1(t) = \dot y(t)
\end{equation}
čo však nie je v tvare aký hľadáme. Na pravej strane vystupuje pôvodná veličina $y(t)$.

Druhou voľbou preto nech je
\begin{equation} \label{volba2}
    x_2(t) = \dot y(t)
\end{equation}
pretože potom môžeme písať prvú diferenciálnu rovnicu v tvare
\begin{equation}
    \dot x_1(t) = x_2(t)
\end{equation}
Ostáva zostaviť druhú diferenciálnu rovnicu. 

Keďže sme zvolili \eqref{volba2}, tak je zrejmé, že platí
\begin{equation} 
    \dot x_2(t) = \ddot y(t)
\end{equation}
Otázkou je $\ddot y(t) = \ ?$ Odpoveďou je pôvodná diferenciálna rovnica druhého rádu. Upravme \eqref{povonadr2} na tvar
\begin{align}
     \ddot y(t) + \frac{a_1}{a_2} \dot y(t) + \frac{a_0}{a_2} y(t) &= \frac{b_0}{a_2} u(t) \\
     \ddot y(t) &= - \frac{a_1}{a_2} \dot y(t) - \frac{a_0}{a_2} y(t) +  \frac{b_0}{a_2} u(t) 
\end{align}
To znamená, že
\begin{equation}  \label{druhadr_1}
    \dot x_2(t) = - \frac{a_1}{a_2} \dot y(t) - \frac{a_0}{a_2} y(t) +  \frac{b_0}{a_2} u(t) 
\end{equation}
čo však stále nie je požadovaný tvar druhej hľadanej diferenciálnej rovnice. Na pravej strane rovnice \eqref{druhadr_1} môžu figurovať len nové veličiny $x_1(t)$ a $x_2(t)$, nie pôvodná veličina $y(t)$. Stačí si však všimnúť skôr zvolené \eqref{volba1} a \eqref{volba2}. Potom môžeme písať
\begin{equation}  \label{druhadr_2}
    \dot x_2(t) = - \frac{a_1}{a_2}  x_2(t) - \frac{a_0}{a_2} x_1(t) +  \frac{b_0}{a_2} u(t) 
\end{equation}
čo je druhá hľadaná diferenciálna rovnica prvého rádu.



Diferenciálnu rovnicu druhého rádu
\begin{equation} 
    a_2 \ddot y(t) + a_1 \dot y(t) + a_0 y(t) = b_0 u(t)
\end{equation}
sme transformovali na sústavu diferenciálnych rovníc prvého rádu
\begin{align}
    \dot x_1(t) &= x_2(t) \\
    \dot x_2(t) &= - \frac{a_1}{a_2}  x_2(t) - \frac{a_0}{a_2} x_1(t) +  \frac{b_0}{a_2} u(t) 
\end{align}






















\section{Analytické riešenie diferenciálnych rovníc}

Riešením dif. rovnice je funkcia času $y(t)$. Ak nájdeme takú funkciu času $y(t)$, ktorú keď dosadíme do dif. rovnice a rovnica bude platiť, potom sme našli riešenie dif. rovnice.


\subsection{Homogénna diferenciálna rovnica prvého rádu}

Majme diferenciálnu rovnicu v tvare:
\begin{equation} \label{hodrpr011}
    \frac{\text{d}y(t)}{\text{d}t} = - a y(t) \qquad y(0)=y_0
\end{equation}
kde $y(t)$ je neznáma funkcia času, riešením rovnice hľadáme túto funkciu času, a $a\in\mathbb{R}$ (je to reálne číslo) je daný parameter. Hodnota $y_0$ v začiatočnej podmienke je tiež daná.



\subsubsection{Náznak metódy separácie premenných}

Upravme diferenciálnu rovnicu \eqref{hodrpr011} tak, aby rovnaké premenné boli na rovnakých stranách. V tvare \eqref{hodrpr011} je signál $y(t)$ na oboch stranách rovnice. Nech je len na ľavej strane. Rovnako, nech čas $t$ je len na pravej strane. Teda
\begin{equation} \label{diffRbeta2}
    \frac{1}{y(t)}\text{d}y(t) = - a \text{d}t
\end{equation}

Všimnime si, že teraz je možné obe strany rovnice integrovať, každú podľa vlastnej premennej, teda
\begin{equation} \label{diffRbeta3}
    \int \frac{1}{y(t)}\text{d}y(t) =  \int - a \text{d}t
\end{equation}
Výsledkom integrovania je
\begin{equation} \label{diffRbeta4}
     \ln \left(  y(t)  \right) + k_1 =   - a t + k_2
\end{equation}
kde $k_1$ a $k_2$ sú konštanty vyplývajúce z neurčitých integrálov (a tiež sme potichu uvážili, že $y(t)$ nebude nadobúdať záporné hodnoty).

Rovnica \eqref{diffRbeta4} už nie je diferenciálna. Žiadna veličina v nej nie je derivovaná podľa času. Vyjadrime z rovnice \eqref{diffRbeta4} signál $y(t)$. Úpravou 
\begin{align}
    \ln \left(  y(t)  \right)  =   - a t + k_3
\end{align}
sme zaviedli konštantu $k_3 = k_2 - k_1$. Ďalej
\begin{subequations}
    \begin{align}
        y(t)   &=  e^{\left( - a t + k_3 \right)} \\
        y(t)   &=  e^{\left( - a t \right)}  e^{k_3} \label{rawRies}
    \end{align}
\end{subequations}

Už v tomto bode je rovnica \eqref{rawRies} predpisom, ktorý udáva časovú závislosť veličiny $y$. Vyjadruje signál (časovú funkciu) $y(t)$. Časová funkcia $y(t)$ je riešením diferenciálnej rovnice \eqref{diffRbeta2}.

V rovnici \eqref{rawRies} je konštanta $e^{k_3}$. Je to všeobecná konštanta a môže mať akúkoľvek hodnotu. Je možné ukázať, my si tu však dovolíme neuviesť formálnu ukážku, že táto konštanta je daná začiatočnou podmienkou priradenou k diferenciálnej rovnici. V~tomto prípade platí $e^{k_3} = y_0$.

Hľadaným riešením diferenciálnej rovnice je časová funkcia v tvare
\begin{align}
    y(t)   &=  y_0 \ e^{\left( - a t \right)}   \label{rawRies2}
\end{align}





\subsubsection{Náznak metódy využívajúcej \emph{charakteristickú rovnicu}}

Pre azda prvý kontakt s pojmom \emph{charakteristická rovnica} uveďme aj náznak podstaty tohto pojmu.

Pre tu uvažovanú diferenciálnu rovnicu, pripomeňme
\begin{equation}
    \dot y(t) + a y(t) = 0
\end{equation}
takpovediac hľadajme riešenie v tvare
\begin{equation}
    y(t) = e^{st}
\end{equation}
kde $s$ je vo všeobecnosti komplexné číslo. Potom platí
\begin{equation}
    \dot y(t) = s\,e^{st}
\end{equation}
Dosadením uvedeného do samotnej rovnice môžme písať
\begin{align}
    s\,e^{st} + a\, e^{st} &= 0 \\
    e^{st} \left(s + a\right) &= 0 
\end{align}
Keďže $e^{st} \neq 0$ (ak to má byť netriviálnym riešením), tak musí platiť
\begin{equation}
    \left(s + a\right) = 0
\end{equation}
Toto je \emph{charakteristická rovnica}.

Je zrejmé, že tu je potrebné nájsť koreň polynómu, nazývajme ho charakteristický polynóm, $s + a$. Koreňom polynómu (riešením charakteristickej rovnice) v tomto prípade je $s_1 = -a$. Týmto sme našli riešenie, ktoré sa nazýva fundamentálne (označme $y_f(t)$), v~tomto prípade:
\begin{equation}
    y_f(t) = e^{\left( - a t \right)} 
\end{equation}
Toto riešenie prislúcha k danému koreňu charakteristickej rovnice. Ako také však nezohľadňuje všetky možné riešenia, ktoré sú pri homogénnej diferenciálnej rovnici dané začiatočnou podmienkou. Všeobecné riešenie by pre tento prípad malo tvar
\begin{equation}
    y(t) = c\,e^{\left( - a t \right)} 
\end{equation}
kde $c$ je konštanta, pomocou ktorej je v tomto prípade zjavne možné stanoviť hodnotu všeobecného riešenia v čase $t=0$ danú začiatočnou podmienkou. Konkrétne riešenie teda je
\begin{align}
    y(t)   &=  y_0 \ e^{\left( - a t \right)}  
\end{align}

Poznámka: v prípade homogénnej rovnice vyššieho rádu by bolo možné ukázať, že všeobecné riešenie je vždy dané ako lineárna kombinácia fundamentálnych riešení.



















\subsection{Metóda charakteristickej rovnice}

Pod týmto názvom budeme v rámci tohto predmetu rozumieť istú metódu pre hľadanie všeobecného riešenia homogénnej diferenciálnej rovnice. Metóda vychádza z analýzy problému, ktorá vedie k stanoveniu istej štruktúry riešení homogénnej diferenciálnej rovnice. Typicky sa daný problém skúma pre prípad dif. rovnice druhého rádu. Nasledujúce je spracované najmä podľa učebného materiálu \cite{Mihalikova2012}.

Zdôraznime prívlastok \emph{homogénna} v zmysle, že predmetom tohto textu je \emph{všeobecné riešenie} dif. rovnice. Takéto riešenie je možné ďalej konkretizovať v zmysle začiatočných (prípadne okrajových) podmienok. 

Uvažujme diferenciálnu rovnicu v tvare
\begin{equation}
    \ddot y(t) + a_1 \dot y(t) + a_0 y(t) = 0 \label{e03}
\end{equation}
kde $y(t)$ je neznáma funkcia času. Túto funkciu hľadáme tak aby bola riešením rovnice \eqref{e03}, teda tak aby po dosadení $y(t)$ a jej derivácií do rovnice \eqref{e03} bola rovnica platná. Pre úplnosť, $a_1 \in \mathbb R$ a $a_2  \in \mathbb R$ sú konštantné koeficienty.

Všeobecné riešenie dif. rovnice \eqref{e03} nájdeme tak, že najprv hľadáme dve lineárne nezávislé riešenia. Tieto riešenia nazývame \emph{fundamentálnymi riešeniami}. Ich lineárna kombinácia je potom všeobecným riešením dif. rovnice \eqref{e03}.



\subsubsection{Fundamentálne riešenia}

Nech $y_1(t)$ a $y_2(t)$ sú dve riešenia dif. rovnice \eqref{e03}. Potom každá ich lineárna kombinácia $c_1 y_1(t) + c_2 y_2(t) $, $c_1, c_2 \in \mathbb R$ je tiež riešením dif. rovnice \eqref{e03} (vyplýva to z~lineárnosti derivácie). 

Špeciálne, ak $c_1 = c_2 = 0$ dostaneme riešenie $y(t) = 0$. Nazýva sa nulovým riešením alebo \emph{triviálnym riešením}.

Ďalším rozborom je možné ukázať, ak sú uvažované dve riešenia $y_1(t)$ a $y_2(t)$ navzájom lineárne nezávislé, potom ich lineárnou kombináciou je možné stanoviť všeobecné riešenie dif. rovnice \cite{Mihalikova2012}. 

Mať jedno riešenie („nejaké riešenie“) a následne sa ho pokúšať zovšeobecniť v~podstate nie je možné. Potrebujeme aspoň dve riešenia (zjavne však nie „hocijaké“) a~potrebujeme aby boli lineárne nezávislé.

S pojmom lineárna nezávislosť funkcií sa viaže pojem \emph{Wronskián} alebo Wronského determinant. 

Dve lineárne nezávislé riešenia dif. rovnice nazývame \emph{fundamentálnymi riešeniami} alebo \emph{bázou riešení} dif. rovnice (pozri tiež \cite{Kuben1995}). Je možné ukázať, že riešenia dif. rovnice tvoria vektorový priestor a ten má vždy bázu \cite{Kuben1995}.

Všeobecným riešením $y(t)$ nazývame lineárnu kombináciu dvoch fundamentálnych riešení $y_{f1}(t)$ a $y_{f2}(t)$, teda $y(t) = c_1 y_{f1}(t) + c_2 y_{f2}(t)$, kde $c_1, c_2 \in \mathbb R$ sú ľubovoľné konštanty.


Nepoznáme všeobecnú metódu, pomocou ktorej by sme vždy vedeli nájsť fundamentálne riešenia. V praxi sa využívajú rôzne postupy, ktoré sú zvyčajne založené na skúšaní rôznych tvarov riešení. Pri lineárnej diferenciálnej rovnici s konštantnými koeficientmi je možné nájsť riešenia v tvare exponenciálnej funkcie. V prípade dif. rovnice \eqref{e03} je možné hľadať riešenie v tvare $y(t) = e^{s t}$, kde $s \in \mathbb C$.




\subsubsection{Všeobecné riešenie homogénnej diferenciálnej rovnice}


Uvažujme diferenciálnu rovnicu v tvare
\begin{equation}
    \ddot y(t) + a_1 \dot y(t) + a_0 y(t) = 0 \label{e01}
\end{equation}
kde $a_1 \in \mathbb R$ a $a_2  \in \mathbb R$.

Riešenie rovnice \eqref{e01} hľadajme v tvare 
\begin{equation}
    y(t) = e^{s t}
\end{equation}
kde $s \in \mathbb C$. Potom
\begin{align}
    \dot y(t) &= s e^{s t} \\
    \ddot y(t) &= s^2 e^{s t}
\end{align}
Dosadením do rovnice \eqref{e01} máme
\begin{subequations}
    \begin{align}
        s^2 e^{s t} + a_1 s e^{s t} + a_0 e^{s t} &= 0 \\
        e^{s t} (s^2 + a_1 s + a_0) &= 0
    \end{align}
\end{subequations}
Keďže $e^{s t} \neq 0$, pretože nehľadáme triviálne riešenie, tak pre nájdenie riešenia musí platiť
\begin{equation}
    s^2 + a_1 s + a_0 = 0 \label{e02}
\end{equation}


\paragraph{Charakteristický polynóm a korene charakteristického polynómu}

Rovnica \eqref{e02} sa nazýva \emph{charakteristická rovnica} dif. rovnice \eqref{e01} a~jej riešenia $s_1$ a~$s_2$ sú \emph{charakteristickými číslami} dif. rovnice \eqref{e01}. Inými slovami polynóm $s^2 + a_1 s + a_0$ sa nazýva \emph{charakteristický polynóm} dif. rovnice \eqref{e01}.

Funkcia $e^{s t}$ je riešením dif. rovnice \eqref{e01} ak $s$ je riešením algebraickej rovnice \eqref{e02}. Inými slovami, $e^{s t}$ je riešením dif. rovnice \eqref{e01} ak $s$ je koreňom charakteristického polynómu dif. rovnice. 


Vo všeobecnosti môžeme určiť toľko navzájom odlišných funkcií $e^{s t}$, teda riešení diferenciálnej rovnice, koľko koreňov má charakteristický polynóm. V tomto prípade dif. rovnice druhého rádu sú to dva korene. Môžu byť
\begin{itemize}
    \item dva navzájom rôzne reálne korene,
    \item jeden dvojnásobný reálny koreň,
    \item dva komplexne združené korene.
\end{itemize}
Tieto prípady vedú k rôznym typom riešení dif. rovnice.







\subsubsection{Prípad: dva navzájom rôzne reálne korene}

Nech charakteristický polynóm má dva navzájom rôzne reálne korene $s_1$ a $s_2$. Potom všeobecné riešenie dif. rovnice \eqref{e01} je
\begin{equation}
    y(t) = c_1 e^{s_1 t} + c_2 e^{s_2 t}
\end{equation}
kde uvažujeme, že fundamentálnymi riešeniami sú
\begin{subequations} \label{kfunr}
    \begin{align}
        y_{f1}(t) &= e^{s_1 t} \\
        y_{f2}(t) &= e^{s_2 t}
    \end{align} 
\end{subequations}
Ich Wronskián (Wronského determinant) \cite{Mihalikova2012, Kuben1995} je
\begin{equation}
    \begin{aligned}
        W(t) &= \begin{vmatrix}
            e^{s_1 t} & e^{s_2 t} \\
            s_1 e^{s_1 t} & s_2 e^{s_2 t}
        \end{vmatrix} \\
        &=e^{s_1t} s_2 e^{s_2t} - e^{s_2} s_1 e^{s_1t} \\
        &= e^{(s_1 + s_2) t} (s_2 - s_1) \neq 0
    \end{aligned}
\end{equation}
čím sme ukázali, že sú lineárne nezávislé a potvrdili, že všeobecné riešenie dif. rovnice \eqref{e01} v tomto prípade je
\begin{equation}
    y(t) = c_1 e^{s_1 t} + c_2 e^{s_2 t}
\end{equation}




\subsubsection{Prípad: jeden dvojnásobný reálny koreň}

Nech charakteristický polynóm má koreň $s$. Riešením dif. rovnice \eqref{e01} je $y_{f1} = e^{st}$. Pre nájdenie všeobecného riešenia však potrebujeme aj druhé fundamentálne riešenie. Je možné ukázať, že $y_{f2} = t e^{st}$ je tiež riešením dif. rovnice \eqref{e01}.
Máme teda dve fundamentálne riešenia v tvare
\begin{subequations}
    \begin{align}
        y_{f1} &= e^{st} \\
        y_{f2} &= t e^{st}
    \end{align}
\end{subequations}
Je možné ukázať, že sú lineárne nezávislé.

Všeobecné riešenie dif. rovnice \eqref{e01} v tomto prípade je v tvare
\begin{equation}
    y(t) = c_1 e^{st} + c_2 t e^{st}
\end{equation}



\subsubsection{Prípad: dva komplexne združené korene}

Nech charakteristický polynóm má dva komplexne združené korene
\begin{subequations}
    \begin{align}
        s_1 &= a + j b \\
        s_2 &= a - j b
    \end{align}
\end{subequations}
kde $a, b \in \mathbb R$ a $j$ je imaginárna jednotka.

V tomto prípade funkcie $e^{(a+jb)t}$ a $e^{(a-jb)t}$ nie sú reálne funkcie. S využitím Eulerovho vzťahu
\begin{subequations}
    \begin{align}
        \begin{split}
            e^{s_1t} = e^{at}e^{jbt} &= e^{at} \left( \cos(bt) + j \sin(bt) \right) \\
            &= e^{at} \cos(bt) + j e^{at} \sin(bt)
        \end{split} \\
        \begin{split}
            e^{s_2t} = e^{at}e^{-jbt} &= e^{at} \left( \cos(bt) - j \sin(bt) \right) \\
            &= e^{at} \cos(bt) - j e^{at} \sin(bt)
        \end{split} 
    \end{align}
\end{subequations}
Uvedené funkcie sú teda lineárnou kombináciou funkcií
\begin{subequations}
    \begin{align}
        y_{f1}(t) &= e^{at} \cos(bt) \\
        y_{f2}(t) &= e^{at} \sin(bt)
    \end{align}
\end{subequations}
Je možné ukázať, že $y_{f1}(t)$ a $y_{f2}(t)$ sú riešeniami dif. rovnice \eqref{e01}, a že sú lineárne nezávislé. Všeobecné riešenie dif. rovnice \eqref{e01} v tomto prípade je v tvare
\begin{equation}
    y(t) = c_1 e^{at} \cos(bt) + c_2 e^{at} \sin(bt)
\end{equation}














\section{Príklady}

\subsection{Príklad 1}

Nájdite analytické riešenie diferenciálnej rovnice. Použite metódu charakteristickej rovnice.
\begin{equation*} 
    \ddot y(t) + 3\dot y(t) + 2 y(t) = 0 \qquad y(0)=3 \qquad \dot y(0) = -2
\end{equation*}

Prvým krokom je stanovenie charakteristickej rovnice. V tomto prípade
\begin{equation}
    s^2 + 3s + 2 = 0
\end{equation}

V druhom kroku pre stanovenie fundamentálnych riešení hľadáme riešenia charakteristickej rovnice. Riešením charakteristickej rovnice sú
\begin{subequations}
    \begin{align}
        s_1 &= -1 \\
        s_2 &= -2
    \end{align}
\end{subequations}
Zodpovedajúce fundamentálne riešenia sú
\begin{subequations}
    \begin{align}
        y_{f1}(t) &= e^{-t} \\
        y_{f2}(t) &= e^{-2t}
    \end{align}
\end{subequations}

Tretím krokom je stanovenie všeobecného riešenia dif. rovnice. Je lineárnou kombináciou fundamentálnych riešení. Teda
\begin{equation}
    y(t) = c_1 e^{-t} + c_2 e^{-2t}
\end{equation}
kde $c_1, c_2 \in \mathbb{R}$ sú konštanty.

Vo štvrtom kroku je možné na základe začiatočných podmienok stanoviť konkrétne riešenie. Pre čas $t = 0$ má všeobecné riešenie tvar
\begin{equation}
    y(0)  = c_1 \, e^{(-1) 0} + c_2 \, e^{(-2) 0} = c_1 + c_2
\end{equation}
Tým sme takpovediac zúžitkovali informáciu o začiatočnej hodnote $y(0) = 3$. Druhá začiatočná podmienka sa týka derivácie riešenia. Derivácia všeobecného riešenia je
\begin{equation}
    \dot y(t) = - c_1 e^{-t} - 2 c_2 e^{-2t}
\end{equation}
Pre čas $t = 0$ má derivácia všeobecného riešenia tvar
\begin{equation}
    \dot y(0)  = - c_1 - 2 c_2
\end{equation}
Z uvedeného vyplýva sústava dvoch rovníc o dvoch neznámych konštantách $c_1$ a $c_2$
\begin{subequations}
    \begin{align}
        c_1 + c_2 &= 3 \\
        - c_1 - 2 c_2 &= -2
    \end{align}
\end{subequations}
Platí $c_2 = 3-c_1$, a teda
\begin{subequations}
    \begin{align}
        - c_1 - 2 (3 - c_1) &= -2 \\
        - c_1 - 6 + 2 c_1 &= -2 \\
        c_1 &= 4
    \end{align}
\end{subequations}
potom
\begin{subequations}
    \begin{align}
        c_2 &= 3 - c_1 \\
        c_2 &= 3 - 4 \\
        c_2 &= -1
    \end{align}
\end{subequations}
Našli sme funkciu $y(t)$, ktorá je riešením diferenciálnej rovnice pre konkrétne začiatočné podmienky
\begin{equation}
    y(t) = 4 e^{-t} - e^{-2t}
\end{equation}





\subsection{Príklad 2}

Nájdite analytické riešenie diferenciálnej rovnice. Použite metódu charakteristickej rovnice.
\begin{equation*} 
    \ddot y(t) + (a+b) \dot y(t) + a b y(t) = 0  \qquad y(0) = y_0 \qquad \dot y(0) = z_0 \qquad a,b \in \mathbb{R}
\end{equation*}

Prvým krokom je stanovenie charakteristickej rovnice. V tomto prípade
\begin{equation}
    s^2 + (a+b) s + a b = 0 
\end{equation}

V druhom kroku pre stanovenie fundamentálnych riešení hľadáme riešenia charakteristickej rovnice. Vo všeobecnosti
\begin{equation}
    s_{1,2} = \frac{-(a + b) \pm \sqrt{(a + b)^2 - 4  ab}}{2}
\end{equation}
avšak v tomto prípade tiež vidíme, že
\begin{equation}
    s^2 + (a+b) s + a b = (s + a)(s + b)
\end{equation}
Riešenia charakteristickej rovnice teda sú
\begin{subequations}
    \begin{align}
        s_1 &= -a \\
        s_2 &= -b
    \end{align}
\end{subequations}
Zodpovedajúce fundamentálne riešenia sú
\begin{subequations}
    \begin{align}
        y_{f1}(t) &= e^{-at} \\
        y_{f2}(t) &= e^{-bt}
    \end{align}
\end{subequations}

Tretím krokom je stanovenie všeobecného riešenia dif. rovnice. Je lineárnou kombináciou fundamentálnych riešení. Teda
\begin{equation}
    y(t) = c_1 e^{-at} + c_2 e^{-bt}
\end{equation}
kde $c_1, c_2 \in \mathbb{R}$ sú konštanty.

Vo štvrtom kroku je možné na základe začiatočných podmienok stanoviť konkrétne riešenie. Pre čas $t = 0$ má všeobecné riešenie tvar
\begin{equation}
    y(0)  = c_1 \, e^{(-a) 0} + c_2 \, e^{(-b) 0} = c_1 + c_2   
\end{equation}
Derivácia všeobecného riešenia je
\begin{equation}
    \dot y(t) = -a c_1 e^{-at} - b c_2 e^{-bt}
\end{equation}
Pre čas $t = 0$ má derivácia všeobecného riešenia tvar
\begin{equation}
    \dot y(0)  = -a c_1 - b c_2
\end{equation}
Z uvedeného vyplýva sústava dvoch rovníc o dvoch neznámych konštantách $c_1$ a $c_2$
\begin{subequations}
    \begin{align}
        c_1 + c_2 &= y_0 \\
        -a c_1 - b c_2 &= z_0
    \end{align}
\end{subequations}
Do druhej rovnice dosaďme $c_1 = y_0 - c_2$
\begin{subequations}
    \begin{align}
        -a (y_0 - c_2) - b c_2 &= z_0 \\
        -a y_0 + a c_2 - b c_2 &= z_0 \\
        c_2 (a - b) &= z_0 + a y_0 \\
        c_2 &= \frac{z_0 + a y_0}{a - b}
    \end{align}
\end{subequations}
potom
\begin{subequations}
    \begin{align}
        c_1 &= y_0 - c_2 \\
        c_1 &= y_0 - \frac{z_0 + a y_0}{a - b} \\  
        c_1 &= \frac{y_0 (a - b) - z_0 - a y_0}{a - b}\\
        c_1 &= \frac{y_0 a - y_0 b - z_0 - a y_0}{a - b}\\
        c_1 &= \frac{- y_0 b - z_0}{a - b}          
    \end{align}
\end{subequations}
Konkrétne riešenie úlohy teda je
\begin{equation}
    y(t) = \frac{- y_0 b - z_0}{a - b} e^{-at} + \frac{z_0 + a y_0}{a - b} e^{-bt}
\end{equation}











\section{Otázky a úlohy}

\begin{enumerate}[leftmargin=0pt, labelsep=3mm, itemsep=0pt]
    
    \item Čo je riešením obyčajnej diferenciálnej rovnice (vo všeobecnosti)?

    \item Vysvetlite rozdiel medzi homogénnou a nehomogénnou obyčajnou diferenciálnou rovnicou.

    \item Uveďte príklad homogénnej obyčajnej diferenciálnej rovnice.

    \item Uveďte príklad nehomogénnej obyčajnej diferenciálnej rovnice.

    \item Vysvetlite pojem \emph{analytické riešenie} obyčajnej diferenciálnej rovnice.
    
    \item Nájdite analytické riešenie diferenciálnej rovnice
    \begin{align*}
        \dot y(t) + a y(t) = 0 \qquad y(0) = y_0 \qquad a\in\mathbb R,\ y_0\in\mathbb R
    \end{align*}    

	\item Nájdite analytické riešenie diferenciálnej rovnice (nejaká bude zadaná).
	
    \item Z diferenciálnej rovnice vyššieho rádu zostavte sústavu diferenciálnych rovníc prvého rádu (bude zadaný konkrétny príklad).
    
    \item Nasledujúcu diferenciálnu rovnicu druhého rádu prepíšte na sústavu diferenciálnych rovníc prvého rádu.
    \begin{align*}
        a_2 \ddot y(t) + a_1 \dot y(t) + a_0 y(t) = b_0 u(t) \qquad a_2, a_1, a_0, b_0 \in\mathbb R
    \end{align*}


\end{enumerate}














\printbibliography[title={Literatúra}]





\end{document}
