\documentclass[a4paper, 10pt, ]{article}

\input{../misc_LaTeX/preamble.tex}

\def\oznacenieCasti{MRS03 - ZS2023}














\begin{document}


\lstset{%
style=mystyle,
rangebeginprefix=\#\#\#\ cellB\ ,%
rangebeginsuffix=\ \#\#\#,%
rangeendprefix=\#\#\#\ cellE\ ,%
rangeendsuffix=\ \#\#\#,%
includerangemarker=false,
}






\fontsize{12pt}{22pt}\selectfont

\centerline{\textsf{Modelovanie a riadenie systémov} \hfill \textsf{\oznacenieCasti}}

\fontsize{18pt}{22pt}\selectfont







\begin{flushleft}
	\textbf{\textsf{Rôzne názorné príklady\\k téme diferenciálne rovnice\\(s RC obvodom)}}
\end{flushleft}





\normalsize

\bigskip

{\hypersetup{hidelinks}

\tableofcontents

}

\bigskip

\vspace{18pt}






\section{Homogénna diferenciálna rovnica prvého rádu}


\subsection{Rovnica, schéma, systém}

Majme diferenciálnu rovnicu v tvare:
\begin{equation} \label{hodrpr01}
    \frac{\text{d}y(t)}{\text{d}t} = - a y(t) \qquad y(0)=y_0
\end{equation}
kde $y(t)$ je neznáma funkcia času, riešením rovnice hľadáme túto funkciu času, a $a\in\mathbb{R}$ (je to reálne číslo) je daný parameter. Hodnota $y_0$ v začiatočnej podmienke je tiež daná.

Ide o homogénnu diferenciálnu rovnicu pretože neobsahuje žiadne iné členy len také, ktoré obsahujú neznámu $y(t)$. Ak by sme členy obsahujúce len neznámu presunuli na ľavú stranu, teda:
\begin{equation}
    \frac{\text{d}y(t)}{\text{d}t} + a y(t) = 0
\end{equation}
na pravej strane je nula. Rovnica je homogénna.






Z hľadiska systémového je $y(t)$ výstupnou veličinou (výstupným signálom) a číslo $a$ je parameter systému. $y(0)=y_0$ je začiatočná podmienka kde $y_0$ je hodnota signálu $y(t)$ v čase $t=0$.

V tomto prípade nemáme vstupnú veličinu systému (vstupný signál). Ako sme uviedli, z pohľadu diferenciálnej rovnice, alebo z hľadiska pojmu dynamický systém, hovoríme o homogénnej diferenciálnej rovnici. Obsahuje len samotnú neznámu (výstupnú veličinu, výstupný signál).

Dôležitým faktom je, že táto rovnica sa nazýva diferenciálna rovnica prvého rádu, pretože najvyššia derivácia neznámej je prvého rádu.


Mimochodom, operáciu derivácie je najmä v inžinierskych súvislostiach značiť pomocou bodky nad signálom, teda napr. $\frac{\text{d}y(t)}{\text{d}t} = \dot y(t)$. Obdobne potom $\frac{\text{d}^2y(t)}{\text{d}t^2} = \ddot y(t)$.


\paragraph{Fyzikálna podstata -- príklad}

Takáto rovnica by mohla opisovať dynamický dej, napríklad vybíjanie kondenzátora cez rezistor. Situáciu je možné znázorniť nasledujúcou elektrickou schémou:


\begin{center}

	\makebox[\textwidth][c]{%
	\input{../fig_standalone/RCclanok_v2.pdf_tex}
	}

    \figcaption{Obvod pre vybíjanie kondenzátora}
	\label{RCclanok_v2}

\end{center}

\noindent
kde $R$ a $C$ sú hodnoty rezistora a kondenzátora. O kondenzátore sa predpokladá, že takpovediac na začiatku času je nabitý, je na ňom isté elektrické napätie.

Časový priebeh veličiny, ktorý je opísaný diferenciálnou rovnicou, ktorou sa tu zaoberáme, je napätie medzi svorkami \textsf{GND} a \textsf{OUT}.



\paragraph{Schematické znázornenie systému}


Diferenciálnu rovnicu, alebo z iného hľadiska dynamický systém v tvare
\begin{equation}  \label{hodrpr01_2}
    \frac{\text{d}y(t)}{\text{d}t} = - a y(t) \qquad y(0)=y_0
\end{equation}
je možné schematicky znázorniť s využitím základných funkčných blokov, stavebných prvkov, medzi ktorými sú signáli (v čase premenlivé veličiny).


\begin{center}

	\makebox[\textwidth][c]{%
	\input{../fig_standalone/RCclanok_v2_sch_v2.pdf_tex}
	}

    \figcaption{Schéma dynamického systému zodpovedajúca rovnici \eqref{hodrpr01_2}.}
	\label{RCclanok_v2_sch_v2}

\end{center}












\subsection{Hľadanie analytického riešenia}


\subsubsection{Náznak metódy separácie premenných}

Upravme diferenciálnu rovnicu \eqref{hodrpr01} tak, aby rovnaké premenné boli na rovnakých stranách. V tvare \eqref{hodrpr01} je signál $y(t)$ na oboch stranách rovnice. Nech je len na ľavej strane. Rovnako, nech čas $t$ je len na pravej strane. Teda
\begin{equation} \label{diffRbeta2}
    \frac{1}{y(t)}\text{d}y(t) = - a \text{d}t
\end{equation}

Všimnime si, že teraz je možné obe strany rovnice integrovať, každú podľa vlastnej premennej, teda
\begin{equation} \label{diffRbeta3}
    \int \frac{1}{y(t)}\text{d}y(t) =  \int - a \text{d}t
\end{equation}
Výsledkom integrovania je
\begin{equation} \label{diffRbeta4}
     \ln \left(  y(t)  \right) + k_1 =   - a t + k_2
\end{equation}
kde $k_1$ a $k_2$ sú konštanty vyplývajúce z neurčitých integrálov (a tiež sme potichu uvážili, že $y(t)$ nebude nadobúdať záporné hodnoty).


Rovnica \eqref{diffRbeta4} už nie je diferenciálna. Žiadna veličina v nej nie je derivovaná podľa času. Vyjadrime z rovnice \eqref{diffRbeta4} signál $y(t)$. Úpravou 
\begin{align}
    \ln \left(  y(t)  \right)  =   - a t + k_3
\end{align}
sme zaviedli konštantu $k_3 = k_2 - k_1$. Ďalej
\begin{subequations}
    \begin{align}
        y(t)   &=  e^{\left( - a t + k_3 \right)} \\
        y(t)   &=  e^{\left( - a t \right)}  e^{k_3} \label{rawRies}
    \end{align}
\end{subequations}

Už v tomto bode je rovnica \eqref{rawRies} predpisom, ktorý udáva časovú závislosť veličiny $y$. Vyjadruje signál (časovú funkciu) $y(t)$. Časová funkcia $y(t)$ je riešením diferenciálnej rovnice \eqref{diffRbeta2}.

V rovnici \eqref{rawRies} je konštanta $e^{k_3}$. Je to všeobecná konštanta a môže mať akúkoľvek hodnotu. Je možné ukázať, my si tu však dovolíme neuviesť formálnu ukážku, že táto konštanta je daná začiatočnou podmienkou priradenou k diferenciálnej rovnici. V~tomto prípade platí $e^{k_3} = y_0$.

Hľadaným riešením diferenciálnej rovnice je časová funkcia v tvare
\begin{align}
    y(t)   &=  y_0 \ e^{\left( - a t \right)}   \label{rawRies2}
\end{align}




\subsubsection{Náznak metódy využívajúcej \emph{charakteristickú rovnicu}}

Pre azda prvý kontakt s pojmom \emph{charakteristická rovnica} uveďme aj náznak podstaty tohto pojmu.

Pre tu uvažovanú diferenciálnu rovnicu, pripomeňme
\begin{equation}
    \dot y(t) + a y(t) = 0
\end{equation}
takpovediac hľadajme riešenie v tvare
\begin{equation}
    y(t) = e^{st}
\end{equation}
kde $s$ je vo všeobecnosti komplexné číslo. Potom platí
\begin{equation}
    \dot y(t) = s\,e^{st}
\end{equation}
Dosadením uvedeného do samotnej rovnice môžme písať
\begin{align}
    s\,e^{st} + a\, e^{st} &= 0 \\
    e^{st} \left(s + a\right) &= 0 
\end{align}
Keďže $e^{st} \neq 0$ (ak to má byť netriviálnym riešením), tak musí platiť
\begin{equation}
    \left(s + a\right) = 0
\end{equation}
Toto je \emph{charakteristická rovnica}.

Je zrejmé, že tu je potrebné nájsť koreň polynómu, nazývajme ho charakteristický polynóm, $s + a$. Koreňom polynómu (riešením charakteristickej rovnice) v tomto prípade je $s_1 = -a$. Týmto sme našli riešenie, ktoré sa nazýva fundamentálne (označme $y_f(t)$), v~tomto prípade:
\begin{equation}
    y_f(t) = e^{\left( - a t \right)} 
\end{equation}
Toto riešenie prislúcha k danému koreňu charakteristickej rovnice. Ako také však nezohľadňuje všetky možné riešenia, ktoré sú pri homogénnej diferenciálnej rovnici dané začiatočnou podmienkou. Všeobecné riešenie by pre tento prípad malo tvar
\begin{equation}
    y(t) = c\,e^{\left( - a t \right)} 
\end{equation}
kde $c$ je konštanta, pomocou ktorej je v tomto prípade zjavne možné stanoviť hodnotu všeobecného riešenia v čase $t=0$ danú začiatočnou podmienkou. Konkrétne riešenie teda je
\begin{align}
    y(t)   &=  y_0 \ e^{\left( - a t \right)}  
\end{align}

Poznámka: v prípade homogénnej rovnice vyššieho rádu by bolo možné ukázať, že všeobecné riešenie je vždy dané ako lineárna kombinácia fundamentálnych riešení.



% \subsection{Hľadanie numerického riešenia}






\section{Nehomogénna diferenciálna rovnica prvého rádu}


\subsection{Rovnica, schéma, systém}

Preskúmajme prípad, ktorý vznikne tak, že v obvode na obr. \ref{RCclanok_v2} budeme uvažovať aj možnosť nabíjať kondenzátor externým napätím (okrem skutočnosti, že kondenzátor môže, ale nemusí, byť nabitý na začiatku).

Situáciu je možné znázorniť nasledujúcou elektrickou schémou:

\begin{center}

	\makebox[\textwidth][c]{%
	\input{../fig_standalone/RCclanok_v3.pdf_tex}
	}

    \figcaption{Obvod pre nabíjanie a vybíjanie kondenzátora}
	\label{RCclanok_v3}

\end{center}

\noindent
kde $R$ a $C$ sú hodnoty rezistora a kondenzátora. O kondenzátore sa predpokladá, že takpovediac na začiatku času nie je nabitý, vo všeobecnosti by však mohol byť. Takémuto obvodu hovoríme štvorpól (spolu štyri svorky, dve vstupné, dve výstupné) a v tomto prípade ide konkrétne o \emph{RC štvorpól} alebo \emph{RC filter} či \emph{RC článok}.

Napätie medzi svorkami \textsf{OUT} a \textsf{GND} nech predstavuje výstupný signál systému a~napätie medzi svorkami \textsf{IN} a \textsf{GND} nech prestavuje vstupný signál systému.


\paragraph{Fyzikálna podstata -- odvodenie diferenciálnej rovnice}


Vstupné napätie (vstupný signál) označme $u(t)$ a výstupné napätie (výstupný signál) označme $y(t)$.

Pre elektrický prúd $I_R(t)$ prechádzajúci rezistorom v tomto prípade platí
\begin{equation}
    I_R(t) = \frac{u(t) - y(t)}{R}
\end{equation}
Pre prúd $I_C(t)$ prechádzajúci kondenzátorom platí
\begin{equation}
    I_C(t) = C \frac{ \text{d}y(t)}{\text{d}t}
\end{equation}
Pritom je zo schémy na obr.~\ref{RCclanok_v3} zrejmé, že
\begin{equation}
    I_C(t) = I_R(t)
\end{equation}
a teda
\begin{align}
    C \dot y(t) &= \frac{u(t) - y(t)}{R} \\
    \dot y(t) &= \frac{u(t) - y(t)}{RC} \\
    \dot y(t) &= -\frac{1}{RC} y(t)  + \frac{1}{RC} u(t)
\end{align}
prípadne ak dáme členy týkajúce sa výstupného signálu (neznámej v rovnici) na ľavú stranu
\begin{align}
    \dot y(t) + \frac{1}{RC} y(t) &=   \frac{1}{RC} u(t)
\end{align}
Toto je nehomogénna diferenciálna rovnica, keďže obsahuje aj inú veličinu (signál)  ako len neznámu $y(t)$.


Pre mierne zovšeobecnenie ďalej uvažujme rovnicu v tvare 
\begin{align}
    \dot y(t) + a y(t) &=   b u(t)
\end{align}
kde $a$ a $b$ sú vo všeobecnosti konštantné koeficienty (sú to parametre uvažovaného systému).

\subsection{Hľadanie analytického riešenia}

\subsubsection{Náznak metódy integračného faktora}

Pre výstupnú veličinu $y(t)$ uvažujme nulovú začiatočnú podmienku $y(0) = 0$. 

Totiž celkové konkrétne riešenie tu uvažovaného problému je vlastne dané dvoma vplyvmi. Vplyv začiatočného stavu výstupnej veličiny a vplyv vstupného signálu. Zložka riešenia daná vlastnou začiatočnou podmienkou by bola získaná rovnako ako keby sme sa na problém pozerali ako na homogénnu diferenciálnu rovnicu. Tým, že uvažujeme $y(0) = 0$ tak začiatočná podmienka nemá žiadny vplyv na výsledné riešenie. Tu sa chceme zamerať na ukážku vplyvu vstupného signálu na riešenie. Avšak, prakticky nie je možné uvažovať akúkoľvek časovú funkciu, ktorá by udávala priebeh signálu $u(t)$. Pre ľubovoľný signál $u(t)$ nemusí byť vždy možné analyticky nájsť riešenie. Je možné nájsť isté triedy časových funkcií vhodných pre reprezentáciu signálu $u(t)$, pre ktoré je následne možné nájsť riešenie diferenciálnej rovnice. Napríklad konštantný signál na vstupe je užitočnou situáciou pre analytické vyšetrovanie dynamických vlastností systému.

Uvažujme konštantný vstupný signál v tvare $u(t) = u_0$, kde $u_0$ daná konštanta (číslo).

Úloha je teda nasledovná
\begin{align}
    \dot y(t) + a y(t) &=   b u(t) \qquad y(0) = 0, \quad u(t) = u_0
\end{align}

S využitím princípu metódy pre riešenie nehomogénnej diferenciálnej funkcie prvého rádu známej ako metóda integračného faktora vynásobme pravú aj ľavú stranu rovnice funkciou $q(t) = e^{at}$ (kde $a$ je koeficient z rovnice), potom
\begin{align}
    \frac{ \text{d}y(t)}{\text{d}t} e^{at} + a y(t) e^{at} &=   b u(t) e^{at}
\end{align}
Funkcia $q(t)$ tu bola skonštruovaná tak aby platilo, že
\begin{align}
    \frac{ \text{d}y(t)}{\text{d}t} e^{at} +  y(t) e^{at} a &=   \frac{ \text{d} \left(y(t)\, e^{at}\right)}{\text{d}t} \\
    \frac{ \text{d}y(t)}{\text{d}t} e^{at} + a y(t) e^{at} &=   \frac{ \text{d} \left(y(t)\, e^{at}\right)}{\text{d}t}
\end{align}
Takže ľavá strana rovnice prejde do tvaru:
\begin{align}
    \frac{ \text{d} \left(y(t)\, e^{at}\right)}{\text{d}t} &= b\, u(t)\, e^{at}
\end{align}
Následne môžme integrovať obe strany rovnice
\begin{align}
    y(t)\, e^{at} &= \int  b\, u(t)\, e^{at}\, \text{d}t 
\end{align}
a upraviť
\begin{align}
    y(t)\, e^{at} &= \int  b\, u_0\, e^{at}\, \text{d}t \\
    y(t)\, e^{at} &=  b\, u_0\, \int   e^{at}\, \text{d}t \\
    y(t)\, e^{at} &=  b\, u_0\,   \left( e^{at} + c \right) \\
    y(t)  &=  b\, u_0\,   \left( e^{at} + c \right)\, e^{-at} \\
    y(t)  &=  b\, u_0\,   \left( 1 + c\, e^{-at} \right) \label{vseobries_inffakt}
\end{align}
Ďalej je potrebné nájsť konštantu $c$ tak aby bola splnená začiatočná podmienka. Časová funkcia \eqref{vseobries_inffakt} je všeobecné riešenie pre akýkoľvek čas $t$. Pre čas $t=0$ potom platí
\begin{align}
    y(0)  &=  b\, u_0\,   \left( 1 + c  \right) \\
    0  &=  b\, u_0 +  b\, u_0 \, c  \\
    - b\, u_0  &=    b\, u_0 \, c  \\
    -\frac{ b\, u_0}{b\, u_0}  &=     c  \\
    c &= -1
\end{align}
Ak je začiatočná podmienka výstupnej veličiny v tomto prípade nulová, pre konštantu vždy platí $c = -1$.

Výsledné riešenie je tak dané, samozrejme parametrami systému, v tomto prípade hodnotami $a$ a $b$, a konštantnou hodnotou vstupného signálu $u(t)$.
\begin{align}
    y(t)  &=  b\, u_0\,   \left( 1 - e^{-at} \right) 
\end{align}





















\end{document}
