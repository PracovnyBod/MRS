\documentclass[a4paper, 10pt, ]{article}

\usepackage[slovak]{babel}





\usepackage[utf8]{inputenc}
\usepackage[T1]{fontenc}

\usepackage[left=4cm,
			right=4cm,
            % left=2.5cm,
			% right=5.5cm,
			top=2.1cm,
			bottom=2.6cm,
			footskip=7.5mm,
			% twoside,
			marginparwidth=3.0cm,
			%showframe,
			]{geometry}

\usepackage{graphicx}
\usepackage[dvipsnames]{xcolor}
% https://en.wikibooks.org/wiki/LaTeX/Colors


% ------------------------------

\usepackage{lmodern}

\usepackage[tt={oldstyle=false,proportional=true,monowidth}]{cfr-lm}

% ------------------------------

\usepackage{amsmath}
\usepackage{amssymb}
\usepackage{amsthm}

\usepackage{booktabs}
\usepackage{multirow}
\usepackage{array}
\usepackage{dcolumn}


\usepackage[singlelinecheck=true]{subfig}


% ------------------------------


\def\naT{\mathsf{T}}

\hyphenpenalty=6000
\tolerance=1000




% ------------------------------


\makeatletter

	\def\@seccntformat#1{\protect\makebox[0pt][r]{\csname the#1\endcsname\hspace{4mm}}}

	\def\cleardoublepage{\clearpage\if@twoside \ifodd\c@page\else
	\hbox{}
	\vspace*{\fill}
	\begin{center}
	\phantom{}
	\end{center}
	\vspace{\fill}
	\thispagestyle{empty}
	\newpage
	\if@twocolumn\hbox{}\newpage\fi\fi\fi}

	\newcommand\figcaption{\def\@captype{figure}\caption}
	\newcommand\tabcaption{\def\@captype{table}\caption}

\makeatother


% ------------------------------




\usepackage{fancyhdr}
\fancypagestyle{plain}{%
\fancyhf{} % clear all header and footer fields
\fancyfoot[C]{\sffamily {\bfseries \thepage}\ | {\scriptsize\oznacenieCasti}}
\renewcommand{\headrulewidth}{0pt}
\renewcommand{\footrulewidth}{0pt}}
\pagestyle{plain}


% ------------------------------


\usepackage{titlesec}
\titleformat{\paragraph}[hang]{\sffamily  \bfseries}{}{0pt}{}
\titlespacing*{\paragraph}{0mm}{3mm}{1mm}
\titlespacing*{\subparagraph}{0mm}{3mm}{1mm}

\titleformat*{\section}{\sffamily\Large\bfseries}
\titleformat*{\subsection}{\sffamily\large\bfseries}
\titleformat*{\subsubsection}{\sffamily\normalsize\bfseries}






% ------------------------------

\PassOptionsToPackage{hyphens}{url}
\usepackage[pdfauthor={},
			pdftitle={},
			pdfsubject={},
			pdfkeywords={},
			% hidelinks,
			colorlinks=false,
			breaklinks,
			]{hyperref}


% ------------------------------


\graphicspath{%
{../fig_standalone/}%
{../../PY/fig/}%
{../../PY/jupynotex/fig/}%
{../../ML/fig/}%
{./fig/}%
}



% ------------------------------

\usepackage{enumitem}

\usepackage{lettrine}

% ------------------------------


\usepackage{microtype}


% ------------------------------

\usepackage[titles]{tocloft}

\setlength{\cftsecindent}{-12mm}
\setlength{\cftsecnumwidth}{12mm}
\renewcommand{\cftsecpresnum}{\hfill}
\renewcommand{\cftsecaftersnum}{\hspace{4mm}}

\setlength{\cftsubsecindent}{-12mm}
\setlength{\cftsubsecnumwidth}{16mm} % 12 + 4
\renewcommand{\cftsubsecpresnum}{\hfill}
\renewcommand{\cftsubsecaftersnum}{\hspace{8mm}} % 4 + 4 mm

\setlength{\cftsubsubsecindent}{-12mm}
\setlength{\cftsubsubsecnumwidth}{20mm} % 12 + 4 + 4
\renewcommand{\cftsubsubsecpresnum}{\hfill}
\renewcommand{\cftsubsubsecaftersnum}{\hspace{12mm}} % 4 + 4 + 4 mm

\renewcommand{\cftsecpagefont}{\lstyle \bfseries}
\renewcommand{\cftsubsecpagefont}{\lstyle}
\renewcommand{\cftsubsubsecpagefont}{\lstyle}



\setlength{\cftparaindent}{-16mm}
\setlength{\cftparanumwidth}{28mm} % 16 + 4 + 4 + 4
\renewcommand{\cftparapresnum}{\hfill}
\renewcommand{\cftparaaftersnum}{\hspace{16mm}} % 4 + 4 + 4 + 4 mm








% ------------------------------

\usepackage{listings}



\renewcommand{\lstlistingname}{Výpis kódu}
\renewcommand{\lstlistlistingname}{Výpisy kódu}




%New colors defined below
\definecolor{codegreen}{rgb}{0,0.6,0}
\definecolor{codegray}{rgb}{0.5,0.5,0.5}
\definecolor{codepurple}{rgb}{0.58,0,0.82}
\definecolor{backcolour}{rgb}{0.95,0.95,0.95}

%Code listing style named "mystyle"
\lstdefinestyle{mystyle}{
  backgroundcolor=\color{backcolour},
  commentstyle=\fontfamily{lmtt}\fontsize{8.5pt}{8.75pt}\selectfont\color{codegreen},
  keywordstyle=\fontfamily{lmtt}\fontsize{8.5pt}{8.75pt}\selectfont\bfseries\color{Blue},
  stringstyle=\fontfamily{lmtt}\fontsize{8.5pt}{8.75pt}\selectfont\color{codepurple},
  basicstyle=\fontfamily{lmtt}\fontsize{8.5pt}{8.75pt}\selectfont,
  breakatwhitespace=false,
  breaklines=true,
  captionpos=t,
  keepspaces=true,
  numbers=left,
  numbersep=4mm,
  numberstyle=\fontfamily{lmtt}\fontsize{8.5pt}{8.75pt}\selectfont\color{lightgray},
  showspaces=false,
  showstringspaces=false,
  showtabs=false,
  tabsize=2,
  % xleftmargin=10pt,
  framesep=10pt,
  language=Python,
  escapechar=|,
}


\lstset{
    inputencoding=utf8,
    extendedchars=true,
    literate=%
    {á}{{\'a}}1
    {č}{{\v{c}}}1
    {ď}{{\v{d}}}1
    {é}{{\'e}}1
    {ě}{{\v{e}}}1
    {í}{{\'i}}1
    {ň}{{\v{n}}}1
    {ó}{{\'o}}1
    {ř}{{\v{r}}}1
    {š}{{\v{s}}}1
    {ť}{{\v{t}}}1
    {ú}{{\'u}}1
    {ů}{{\r{u}}}1
    {ý}{{\'y}}1
    {ž}{{\v{z}}}1
    {Á}{{\'A}}1
    {Č}{{\v{C}}}1
    {Ď}{{\v{D}}}1
    {É}{{\'E}}1
    {Ě}{{\v{E}}}1
    {Í}{{\'I}}1
    {Ň}{{\v{N}}}1
    {Ó}{{\'O}}1
    {Ř}{{\v{R}}}1
    {Š}{{\v{S}}}1
    {Ť}{{\v{T}}}1
    {Ú}{{\'U}}1
    {Ů}{{\r{U}}}1
    {Ý}{{\'Y}}1
    {Ž}{{\v{Z}}}1
    {ô}{{\^{o}}}1
}


% ------------------------------


\usepackage{caption}

\DeclareCaptionFormat{odsadene}{\protect\makebox[0pt][r]{#1#2\hspace{4mm}}#3\par}
\DeclareCaptionLabelSeparator{lendvojbodka}{:}
% \DeclareCaptionFont{lightgray}{\color{lightgray}}
\DeclareCaptionFont{lightgray}{\fontfamily{lmtt}\fontsize{8.5pt}{8.75pt}\selectfont\color{lightgray}}

\captionsetup[lstlisting]{format=odsadene, labelsep=lendvojbodka, justification=raggedright, singlelinecheck=false, labelfont={sf, lightgray},}


% ------------------------------





% ------------------------------

\usepackage[backend=biber,
            style=numeric,
            sorting=none,
            ]{biblatex}
\DeclareSourcemap{
    \maps[datatype=bibtex]{
        \map{
        \step[fieldset=note, null]
        }
        \map{
        \step[fieldset=file, null]
        }        
        % \map{
        % \step[fieldset=url, null]        
        % }
        \map{
        \step[fieldset=eprint, null]
        }
    }
}


\addbibresource{E:/_CurrentContent/01_work_repo/bibLaTeXDB/bibLaTeXDB.bib} % nonpublic data





\def\oznacenieCasti{MRS09 - ZS2025}





\begin{document}


\lstset{%
style=mystyle,
rangebeginprefix=\#\#\#\ cellB\ ,%
rangebeginsuffix=\ \#\#\#,%
rangeendprefix=\#\#\#\ cellE\ ,%
rangeendsuffix=\ \#\#\#,%
includerangemarker=false,
}






\fontsize{12pt}{22pt}\selectfont

\centerline{\textsf{Modelovanie a riadenie systémov} \hfill \textsf{\oznacenieCasti}}

\fontsize{18pt}{22pt}\selectfont





\begin{flushleft}
	\textbf{\textsf{Cvičenie šieste}}
\end{flushleft}






\normalsize

\bigskip

{\hypersetup{hidelinks}

\tableofcontents

}

\bigskip

\vspace{18pt}



\noindent
\lettrine[lines=3, nindent=0pt]{C}{ieľom} cvičení sú témy primárne týkajúce sa analytického riešenia diferenciálnych rovníc s využitím Laplaceovej transformácie. K tomu je možné sledovať súvislosti z hľadiska pojmu \emph{prenosová funkcia}.




\section{Analytické riešenie dif. rovnice s využitím Laplaceovej transformácie}


\subsection{Príklad 1 --  homogénna DR1R}

Nájdite analytické riešenie diferenciálnej rovnice s využitím Laplaceovej transformácie. Rovnica je tvare
\begin{align}
    \dot y(t) - a y(t) = 0 \qquad y(0) = y_0
\end{align}
kde $a \in \mathbb R$ je parameter a $y_0$ je hodnota začiatočnej podmienky. 




\subsection{Príklad 2 --  nehomogénna DR1R (vstup je Dirackov impulz)}
\label{s1pr2}

Nájdite analytické riešenie diferenciálnej rovnice s využitím Laplaceovej transformácie. Rovnica je tvare
\begin{align}
    \dot y(t) - a y(t) = u(t) \qquad y(0) = 0
\end{align}
kde $a \in \mathbb R$ je parameter a $u(t) = \delta(t)$, čo je Dirackov impulz v čase $0$.



\subsection{Postreh k príkladu 1 a 2}

Riešenie systému prvého rádu (čo je len „slang“ pre riešenie diferenciálnej rovnice prvého rádu opisujúcej dynamický systém), ktorý má nulový (žiadny) vstup ale má nenulovú začiatočnú podmienku je, ako vieme, $y(t) = e^{at} y(0)$.

Riešenie systému prvého rádu, ktorý má nulovú začiatočnú podmienku ale na vstupe má Dirackov impulz (signál $\delta(t)$) je $y(t) = e^{at}$.

Ak by sme zovšeobecnili vstup spôsobom $u(t) = y_0 \delta(t)$, teda zaviedli sme faktor $y_0$, potom riešenie je zjavne $y(t) = e^{at} y_0 $. 

Začiatočná podmienka $y(0)$ a faktor $y_0$ plnia tú istú úlohu. Čo znamená, že pomocou Dirackovho impulzu je možné nahradiť vplyv začiatočnej podmienky systému. Nevýhodou je, že v praxi nie je možné realizovať Dirackov impulz, keďže má nekonečne malú šírku, len jeho aproximáciu, ktorou je impulz s relatívne malou šírkou.




\subsection{Príklad 3 --  nehomogénna DR1R (vstup je jednotkový skok)}
\label{s1pr3}

Nájdite analytické riešenie diferenciálnej rovnice s využitím Laplaceovej transformácie. Rovnica je tvare
\begin{align}
    \dot y(t) - a y(t) = u(t) \quad y(0) = 0
\end{align}
kde $a \in \mathbb R$ je parameter a $u(t) = 1$.




\subsection{Príklad 4 --  nehomogénna DR2R}

Nájdite analytické riešenie diferenciálnej rovnice. Použite Laplaceovu transformáciu.
\begin{equation} 
    \ddot y(t) + 4 \dot y(t) + 3y(t) = u(t)    \qquad y(0) = 3 \qquad \dot y(0) = -2  \qquad u(t) = 1 
\end{equation}




\section{Prenosové funkcie}


\subsection{Príklad 1 -- prepis DR na prenosovú funkciu}

Dynamický systém je opísaný diferenciálnou rovnicou
\begin{align}
    \dot y(t) - a y(t) = u(t) 
\end{align}
zapíšte v tvare prenosovej funkcie.





\subsection{Príklad 2 -- prepis DR na prenosovú funkciu}

Dynamický systém je opísaný diferenciálnou rovnicou
\begin{align}
    \ddot y(t) + 4 \dot y(t) + 3y(t) = u(t)
\end{align}
zapíšte v tvare prenosovej funkcie.





\section{Control System Toolbox}

V tejto časti budeme pre výpočty s dynamickými systémami využívať \texttt{Control System Toolbox} v prostredí MATLAB.





\subsection{Príklad 1 (\lstinline|tf|)}

Majme prenosovú funkcie v tvare:
\begin{align}
    G(s) = \frac{1}{s+1}
\end{align}
Vytvorte v MATLABe objekt, ktorý (v rámci \texttt{Control System Toolbox}) reprezentuje túto prenosovú funkciu. Použite príkaz \lstinline|tf|.

\vbox{%
\begin{lstlisting}[language=Matlab, ]
G1 = tf(1,[1 1])
\end{lstlisting}
}

\noindent
Alternatívne:

\vbox{%
\begin{lstlisting}[language=Matlab, ]
s = tf('s');
G2 = 1/(s+1)
\end{lstlisting}
}




\subsection{Príklad 2 (\lstinline|tf|)}

Majme prenosovú funkcie v tvare:
\begin{align}
    G(s) = \frac{s+5}{s^2 + 2 s + 1}
\end{align}
Vytvorte v MATLABe objekt, ktorý (v rámci \texttt{Control System Toolbox}) reprezentuje túto prenosovú funkciu. Použite príkaz \lstinline|tf|.

\vbox{%
\begin{lstlisting}[language=Matlab, ]
G3 = tf([1 5],[1 2 1])
\end{lstlisting}
}

\noindent
Alternatívne:

\vbox{%
\begin{lstlisting}[language=Matlab, ]
s = tf('s');
G4 = (s+5)/(s^2 + 2*s + 1)
\end{lstlisting}
}





\subsection{Príklad 3 (\lstinline|impulse|)}

Príkaz \lstinline|impulse| slúži na nájdenie odozvy systému na Dirackov impulz. Uvažujme systém daný diferenciálnou rovnicou v tvare 
\begin{align} \label{pr3dr}
    \dot y(t) - a y(t) = u(t) 
\end{align}
kde $a = 3$. Stanovte prenosovú funkciu systému. Pomocou príkazu \lstinline|tf| vytvorte objekt reprezentujúci túto prenosovú funkciu. Následne pomocou príkazu \lstinline|impulse| zistite odozvu systému na Dirackov impulz. 

\subsubsection{Dodatok k príkladu 3}
Stanovte časovú funkciu, ktorá je analytickým vyjadrením impulznej charakteristiky systému daného diferenciálnou rovnicou \eqref{pr3dr}, kde $a = 3$. Poznámka: všimnite si úlohu v~časti \ref{s1pr2}. Bonus: graficky porovnajte výsledok príkazu \lstinline|impulse| s analytickým riešením.





\subsection{Príklad 4 (\lstinline|step|)}

Príkaz \lstinline|step| slúži na nájdenie odozvy systému na jednotkový skok. Uvažujme systém daný diferenciálnou rovnicou v tvare 
\begin{align} \label{pr4dr}
    \dot y(t) - a y(t) = u(t) 
\end{align}
kde $a = 3$. Stanovte prenosovú funkciu systému. Pomocou príkazu \lstinline|tf| vytvorte objekt reprezentujúci túto prenosovú funkciu. Následne pomocou príkazu \lstinline|step| zistite odozvu systému na jednotkový skok. 

\subsubsection{Dodatok k príkladu 4}
Stanovte časovú funkciu, ktorá je analytickým vyjadrením prechodovej charakteristiky systému daného diferenciálnou rovnicou \eqref{pr4dr}, kde $a = 3$. Poznámka: všimnite si úlohu v~časti \ref{s1pr3}. Bonus: graficky porovnajte výsledok príkazu \lstinline|step| s analytickým riešením.
















\end{document}
