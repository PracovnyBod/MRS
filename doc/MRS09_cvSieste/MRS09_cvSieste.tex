\documentclass[a4paper, 10pt, ]{article}

\input{../misc_LaTeX/preamble.tex}

\def\oznacenieCasti{MRS09 - ZS2025}





\begin{document}


\lstset{%
style=mystyle,
rangebeginprefix=\#\#\#\ cellB\ ,%
rangebeginsuffix=\ \#\#\#,%
rangeendprefix=\#\#\#\ cellE\ ,%
rangeendsuffix=\ \#\#\#,%
includerangemarker=false,
}






\fontsize{12pt}{22pt}\selectfont

\centerline{\textsf{Modelovanie a riadenie systémov} \hfill \textsf{\oznacenieCasti}}

\fontsize{18pt}{22pt}\selectfont





\begin{flushleft}
	\textbf{\textsf{Cvičenie šieste}}
\end{flushleft}






\normalsize

\bigskip

{\hypersetup{hidelinks}

\tableofcontents

}

\bigskip

\vspace{18pt}



\noindent
\lettrine[lines=3, nindent=0pt]{C}{ieľom} cvičení sú témy primárne týkajúce sa analytického riešenia diferenciálnych rovníc s využitím Laplaceovej transformácie. K tomu je možné sledovať súvislosti z hľadiska pojmu \emph{prenosová funkcia}.




\section{Analytické riešenie dif. rovnice s využitím Laplaceovej transformácie}


\subsection{Príklad 1 --  homogénna DR1R}

Nájdite analytické riešenie diferenciálnej rovnice s využitím Laplaceovej transformácie. Rovnica je tvare
\begin{align}
    \dot y(t) - a y(t) = 0 \qquad y(0) = y_0
\end{align}
kde $a \in \mathbb R$ je parameter a $y_0$ je hodnota začiatočnej podmienky. 




\subsection{Príklad 2 --  nehomogénna DR1R (vstup je Dirackov impulz)}
\label{s1pr2}

Nájdite analytické riešenie diferenciálnej rovnice s využitím Laplaceovej transformácie. Rovnica je tvare
\begin{align}
    \dot y(t) - a y(t) = u(t) \qquad y(0) = 0
\end{align}
kde $a \in \mathbb R$ je parameter a $u(t) = \delta(t)$, čo je Dirackov impulz v čase $0$.



\subsection{Postreh k príkladu 1 a 2}

Riešenie systému prvého rádu (čo je len „slang“ pre riešenie diferenciálnej rovnice prvého rádu opisujúcej dynamický systém), ktorý má nulový (žiadny) vstup ale má nenulovú začiatočnú podmienku je, ako vieme, $y(t) = e^{at} y(0)$.

Riešenie systému prvého rádu, ktorý má nulovú začiatočnú podmienku ale na vstupe má Dirackov impulz (signál $\delta(t)$) je $y(t) = e^{at}$.

Ak by sme zovšeobecnili vstup spôsobom $u(t) = y_0 \delta(t)$, teda zaviedli sme faktor $y_0$, potom riešenie je zjavne $y(t) = e^{at} y_0 $. 

Začiatočná podmienka $y(0)$ a faktor $y_0$ plnia tú istú úlohu. Čo znamená, že pomocou Dirackovho impulzu je možné nahradiť vplyv začiatočnej podmienky systému. Nevýhodou je, že v praxi nie je možné realizovať Dirackov impulz, keďže má nekonečne malú šírku, len jeho aproximáciu, ktorou je impulz s relatívne malou šírkou.




\subsection{Príklad 3 --  nehomogénna DR1R (vstup je jednotkový skok)}
\label{s1pr3}

Nájdite analytické riešenie diferenciálnej rovnice s využitím Laplaceovej transformácie. Rovnica je tvare
\begin{align}
    \dot y(t) - a y(t) = u(t) \quad y(0) = 0
\end{align}
kde $a \in \mathbb R$ je parameter a $u(t) = 1$.




\subsection{Príklad 4 --  nehomogénna DR2R}

Nájdite analytické riešenie diferenciálnej rovnice. Použite Laplaceovu transformáciu.
\begin{equation} 
    \ddot y(t) + 4 \dot y(t) + 3y(t) = u(t)    \qquad y(0) = 3 \qquad \dot y(0) = -2  \qquad u(t) = 1 
\end{equation}




\section{Prenosové funkcie}


\subsection{Príklad 1 -- prepis DR na prenosovú funkciu}

Dynamický systém je opísaný diferenciálnou rovnicou
\begin{align}
    \dot y(t) - a y(t) = u(t) 
\end{align}
zapíšte v tvare prenosovej funkcie.





\subsection{Príklad 2 -- prepis DR na prenosovú funkciu}

Dynamický systém je opísaný diferenciálnou rovnicou
\begin{align}
    \ddot y(t) + 4 \dot y(t) + 3y(t) = u(t)
\end{align}
zapíšte v tvare prenosovej funkcie.





\section{Control System Toolbox}

V tejto časti budeme pre výpočty s dynamickými systémami využívať \texttt{Control System Toolbox} v prostredí MATLAB.





\subsection{Príklad 1 (\lstinline|tf|)}

Majme prenosovú funkcie v tvare:
\begin{align}
    G(s) = \frac{1}{s+1}
\end{align}
Vytvorte v MATLABe objekt, ktorý (v rámci \texttt{Control System Toolbox}) reprezentuje túto prenosovú funkciu. Použite príkaz \lstinline|tf|.

\vbox{%
\begin{lstlisting}[language=Matlab, ]
G1 = tf(1,[1 1])
\end{lstlisting}
}

\noindent
Alternatívne:

\vbox{%
\begin{lstlisting}[language=Matlab, ]
s = tf('s');
G2 = 1/(s+1)
\end{lstlisting}
}




\subsection{Príklad 2 (\lstinline|tf|)}

Majme prenosovú funkcie v tvare:
\begin{align}
    G(s) = \frac{s+5}{s^2 + 2 s + 1}
\end{align}
Vytvorte v MATLABe objekt, ktorý (v rámci \texttt{Control System Toolbox}) reprezentuje túto prenosovú funkciu. Použite príkaz \lstinline|tf|.

\vbox{%
\begin{lstlisting}[language=Matlab, ]
G3 = tf([1 5],[1 2 1])
\end{lstlisting}
}

\noindent
Alternatívne:

\vbox{%
\begin{lstlisting}[language=Matlab, ]
s = tf('s');
G4 = (s+5)/(s^2 + 2*s + 1)
\end{lstlisting}
}





\subsection{Príklad 3 (\lstinline|impulse|)}

Príkaz \lstinline|impulse| slúži na nájdenie odozvy systému na Dirackov impulz. Uvažujme systém daný diferenciálnou rovnicou v tvare 
\begin{align} \label{pr3dr}
    \dot y(t) - a y(t) = u(t) 
\end{align}
kde $a = 3$. Stanovte prenosovú funkciu systému. Pomocou príkazu \lstinline|tf| vytvorte objekt reprezentujúci túto prenosovú funkciu. Následne pomocou príkazu \lstinline|impulse| zistite odozvu systému na Dirackov impulz. 

\subsubsection{Dodatok k príkladu 3}
Stanovte časovú funkciu, ktorá je analytickým vyjadrením impulznej charakteristiky systému daného diferenciálnou rovnicou \eqref{pr3dr}, kde $a = 3$. Poznámka: všimnite si úlohu v~časti \ref{s1pr2}. Bonus: graficky porovnajte výsledok príkazu \lstinline|impulse| s analytickým riešením.





\subsection{Príklad 4 (\lstinline|step|)}

Príkaz \lstinline|step| slúži na nájdenie odozvy systému na jednotkový skok. Uvažujme systém daný diferenciálnou rovnicou v tvare 
\begin{align} \label{pr4dr}
    \dot y(t) - a y(t) = u(t) 
\end{align}
kde $a = 3$. Stanovte prenosovú funkciu systému. Pomocou príkazu \lstinline|tf| vytvorte objekt reprezentujúci túto prenosovú funkciu. Následne pomocou príkazu \lstinline|step| zistite odozvu systému na jednotkový skok. 

\subsubsection{Dodatok k príkladu 4}
Stanovte časovú funkciu, ktorá je analytickým vyjadrením prechodovej charakteristiky systému daného diferenciálnou rovnicou \eqref{pr4dr}, kde $a = 3$. Poznámka: všimnite si úlohu v~časti \ref{s1pr3}. Bonus: graficky porovnajte výsledok príkazu \lstinline|step| s analytickým riešením.
















\end{document}
