\documentclass[a4paper, 10pt, ]{article}

\usepackage[slovak]{babel}





\usepackage[utf8]{inputenc}
\usepackage[T1]{fontenc}

\usepackage[left=4cm,
			right=4cm,
            % left=2.5cm,
			% right=5.5cm,
			top=2.1cm,
			bottom=2.6cm,
			footskip=7.5mm,
			% twoside,
			marginparwidth=3.0cm,
			%showframe,
			]{geometry}

\usepackage{graphicx}
\usepackage[dvipsnames]{xcolor}
% https://en.wikibooks.org/wiki/LaTeX/Colors


% ------------------------------

\usepackage{lmodern}

\usepackage[tt={oldstyle=false,proportional=true,monowidth}]{cfr-lm}

% ------------------------------

\usepackage{amsmath}
\usepackage{amssymb}
\usepackage{amsthm}

\usepackage{booktabs}
\usepackage{multirow}
\usepackage{array}
\usepackage{dcolumn}


\usepackage[singlelinecheck=true]{subfig}


% ------------------------------


\def\naT{\mathsf{T}}

\hyphenpenalty=6000
\tolerance=1000




% ------------------------------


\makeatletter

	\def\@seccntformat#1{\protect\makebox[0pt][r]{\csname the#1\endcsname\hspace{4mm}}}

	\def\cleardoublepage{\clearpage\if@twoside \ifodd\c@page\else
	\hbox{}
	\vspace*{\fill}
	\begin{center}
	\phantom{}
	\end{center}
	\vspace{\fill}
	\thispagestyle{empty}
	\newpage
	\if@twocolumn\hbox{}\newpage\fi\fi\fi}

	\newcommand\figcaption{\def\@captype{figure}\caption}
	\newcommand\tabcaption{\def\@captype{table}\caption}

\makeatother


% ------------------------------




\usepackage{fancyhdr}
\fancypagestyle{plain}{%
\fancyhf{} % clear all header and footer fields
\fancyfoot[C]{\sffamily {\bfseries \thepage}\ | {\scriptsize\oznacenieCasti}}
\renewcommand{\headrulewidth}{0pt}
\renewcommand{\footrulewidth}{0pt}}
\pagestyle{plain}


% ------------------------------


\usepackage{titlesec}
\titleformat{\paragraph}[hang]{\sffamily  \bfseries}{}{0pt}{}
\titlespacing*{\paragraph}{0mm}{3mm}{1mm}
\titlespacing*{\subparagraph}{0mm}{3mm}{1mm}

\titleformat*{\section}{\sffamily\Large\bfseries}
\titleformat*{\subsection}{\sffamily\large\bfseries}
\titleformat*{\subsubsection}{\sffamily\normalsize\bfseries}






% ------------------------------

\PassOptionsToPackage{hyphens}{url}
\usepackage[pdfauthor={},
			pdftitle={},
			pdfsubject={},
			pdfkeywords={},
			% hidelinks,
			colorlinks=false,
			breaklinks,
			]{hyperref}


% ------------------------------


\graphicspath{%
{../fig_standalone/}%
{../../PY/fig/}%
{../../PY/jupynotex/fig/}%
{../../ML/fig/}%
{./fig/}%
}



% ------------------------------

\usepackage{enumitem}

\usepackage{lettrine}

% ------------------------------


\usepackage{microtype}


% ------------------------------

\usepackage[titles]{tocloft}

\setlength{\cftsecindent}{-12mm}
\setlength{\cftsecnumwidth}{12mm}
\renewcommand{\cftsecpresnum}{\hfill}
\renewcommand{\cftsecaftersnum}{\hspace{4mm}}

\setlength{\cftsubsecindent}{-12mm}
\setlength{\cftsubsecnumwidth}{16mm} % 12 + 4
\renewcommand{\cftsubsecpresnum}{\hfill}
\renewcommand{\cftsubsecaftersnum}{\hspace{8mm}} % 4 + 4 mm

\setlength{\cftsubsubsecindent}{-12mm}
\setlength{\cftsubsubsecnumwidth}{20mm} % 12 + 4 + 4
\renewcommand{\cftsubsubsecpresnum}{\hfill}
\renewcommand{\cftsubsubsecaftersnum}{\hspace{12mm}} % 4 + 4 + 4 mm

\renewcommand{\cftsecpagefont}{\lstyle \bfseries}
\renewcommand{\cftsubsecpagefont}{\lstyle}
\renewcommand{\cftsubsubsecpagefont}{\lstyle}



\setlength{\cftparaindent}{-16mm}
\setlength{\cftparanumwidth}{28mm} % 16 + 4 + 4 + 4
\renewcommand{\cftparapresnum}{\hfill}
\renewcommand{\cftparaaftersnum}{\hspace{16mm}} % 4 + 4 + 4 + 4 mm








% ------------------------------

\usepackage{listings}



\renewcommand{\lstlistingname}{Výpis kódu}
\renewcommand{\lstlistlistingname}{Výpisy kódu}




%New colors defined below
\definecolor{codegreen}{rgb}{0,0.6,0}
\definecolor{codegray}{rgb}{0.5,0.5,0.5}
\definecolor{codepurple}{rgb}{0.58,0,0.82}
\definecolor{backcolour}{rgb}{0.95,0.95,0.95}

%Code listing style named "mystyle"
\lstdefinestyle{mystyle}{
  backgroundcolor=\color{backcolour},
  commentstyle=\fontfamily{lmtt}\fontsize{8.5pt}{8.75pt}\selectfont\color{codegreen},
  keywordstyle=\fontfamily{lmtt}\fontsize{8.5pt}{8.75pt}\selectfont\bfseries\color{Blue},
  stringstyle=\fontfamily{lmtt}\fontsize{8.5pt}{8.75pt}\selectfont\color{codepurple},
  basicstyle=\fontfamily{lmtt}\fontsize{8.5pt}{8.75pt}\selectfont,
  breakatwhitespace=false,
  breaklines=true,
  captionpos=t,
  keepspaces=true,
  numbers=left,
  numbersep=4mm,
  numberstyle=\fontfamily{lmtt}\fontsize{8.5pt}{8.75pt}\selectfont\color{lightgray},
  showspaces=false,
  showstringspaces=false,
  showtabs=false,
  tabsize=2,
  % xleftmargin=10pt,
  framesep=10pt,
  language=Python,
  escapechar=|,
}


\lstset{
    inputencoding=utf8,
    extendedchars=true,
    literate=%
    {á}{{\'a}}1
    {č}{{\v{c}}}1
    {ď}{{\v{d}}}1
    {é}{{\'e}}1
    {ě}{{\v{e}}}1
    {í}{{\'i}}1
    {ň}{{\v{n}}}1
    {ó}{{\'o}}1
    {ř}{{\v{r}}}1
    {š}{{\v{s}}}1
    {ť}{{\v{t}}}1
    {ú}{{\'u}}1
    {ů}{{\r{u}}}1
    {ý}{{\'y}}1
    {ž}{{\v{z}}}1
    {Á}{{\'A}}1
    {Č}{{\v{C}}}1
    {Ď}{{\v{D}}}1
    {É}{{\'E}}1
    {Ě}{{\v{E}}}1
    {Í}{{\'I}}1
    {Ň}{{\v{N}}}1
    {Ó}{{\'O}}1
    {Ř}{{\v{R}}}1
    {Š}{{\v{S}}}1
    {Ť}{{\v{T}}}1
    {Ú}{{\'U}}1
    {Ů}{{\r{U}}}1
    {Ý}{{\'Y}}1
    {Ž}{{\v{Z}}}1
    {ô}{{\^{o}}}1
}


% ------------------------------


\usepackage{caption}

\DeclareCaptionFormat{odsadene}{\protect\makebox[0pt][r]{#1#2\hspace{4mm}}#3\par}
\DeclareCaptionLabelSeparator{lendvojbodka}{:}
% \DeclareCaptionFont{lightgray}{\color{lightgray}}
\DeclareCaptionFont{lightgray}{\fontfamily{lmtt}\fontsize{8.5pt}{8.75pt}\selectfont\color{lightgray}}

\captionsetup[lstlisting]{format=odsadene, labelsep=lendvojbodka, justification=raggedright, singlelinecheck=false, labelfont={sf, lightgray},}


% ------------------------------





% ------------------------------

\usepackage[backend=biber,
            style=numeric,
            sorting=none,
            ]{biblatex}
\DeclareSourcemap{
    \maps[datatype=bibtex]{
        \map{
        \step[fieldset=note, null]
        }
        \map{
        \step[fieldset=file, null]
        }        
        % \map{
        % \step[fieldset=url, null]        
        % }
        \map{
        \step[fieldset=eprint, null]
        }
    }
}


\addbibresource{E:/_CurrentContent/01_work_repo/bibLaTeXDB/bibLaTeXDB.bib} % nonpublic data





\def\oznacenieCasti{MRS06 - ZS2024}


\usepackage{longtable}




\begin{document}


\lstset{%
style=mystyle,
rangebeginprefix=\#\#\#\ cellB\ ,%
rangebeginsuffix=\ \#\#\#,%
rangeendprefix=\#\#\#\ cellE\ ,%
rangeendsuffix=\ \#\#\#,%
includerangemarker=false,
}





\fontsize{12pt}{22pt}\selectfont

\centerline{\textsf{Modelovanie a riadenie systémov} \hfill \textsf{\oznacenieCasti}}

\fontsize{18pt}{22pt}\selectfont





\begin{flushleft}
	\textbf{\textsf{Laplaceova transformácia a prenosové funkcie}}
\end{flushleft}





\normalsize

\bigskip

{\hypersetup{hidelinks}

\tableofcontents

}

\bigskip

\vspace{18pt}



\noindent
\lettrine[lines=3, nindent=0pt]{L}{aplaceova} transformácia umožňuje efektívne pracovať s lineárnymi dynamickými systémami. Transformuje a tým zjednodušuje operácie súvisiace s hľadaním riešenia lineárnych diferenciálnych rovníc (LDR). Predovšetkým zjednodušuje prácu s konvolučnou rovnicou (konvolučným integrálom) (pozri časť~\ref{predchcasttato}).

K využitiu Laplaceovej transformácie pri riešení diferenciálnych rovníc patrí aj pojem \emph{prenosová funkcia}. Ak hovoríme o prenosových funkciách, hovoríme o nástroji, ktorý umožňuje analyticky pracovať s dynamickými systémami. V tomto texte však nie je cieľom priamo sa zaoberať prenosovými funkciami. Ide o širší pojem, prípadne samostatný nástroj, ktorý sa netýka len samotného riešenia diferenciálnych rovníc.




\section{Definícia Laplaceovej transformácie}

V hrubých črtách je možné o definícii Laplaceovej transformácie uviesť nasledovné.

Majme časovú funkciu $f(t)$ (s vhodnými vlastnosťami, ktoré tu nebudeme uvádzať). Laplaceova transformácia (LT) transformuje, či mapuje, túto funkciu na inú funkciu. Inú funkciu označme $F(s)$. LT je definovaná podľa vzťahu
\begin{align}
    F(s) = \int_0^\infty f(t) e^{-st}\text{d}t
\end{align}
kde $s$ je komplexná premenná (komplexné číslo).


Hovoríme, že ide o transformáciu z časovej oblasti (domény) do domény komplexnej premennej $s$. Premenná $s$ sa často nazýva aj Laplaceov operátor (súvislosti sa ukážu neskôr). Keďže $s = \sigma + j\omega$ a teda $e^{-(\sigma + j\omega)t}$ je signál obsahujúci vo všeobecnosti aj harmonickú (kmitavú) zložku, v tejto súvislosti hovoríme tiež, že pri LT ide o~transformáciu z časovej oblasti do frekvenčnej oblasti.

Výslednej transformovanej funkcii $F(s)$ sa hovorí tiež \emph{obraz} pôvodného signálu $f(t)$ (alebo Laplaceov obraz signálu).

LT je lineárna transformácia, t.j. ak by sme chceli transformovať súčet dvoch signálov (dvoch časových funkcií) $f(t) + g(t)$ ako celok, tak je to možné urobiť transformáciou signálov jednotlivo a až následne sčítať transformované funkcie $F(s) + G(s)$.



\section{Laplaceove obrazy signálov}

Majme signál $f(t)$. Laplaceovym obrazom (L-obrazom) tohto signálu je $F(s)$ (samozrejme v zmysle definície LT) a samotnú operáciu transformácie značíme ako
\begin{align}
    F(s)  =  \mathcal L \left\{ f(t) \right\} = \int_0^\infty f(t) e^{-st}\text{d}t
\end{align}


\subsection{Derivácia}


Nájdime L-obraz signálu $\frac{\text{d}f(t)}{dt}$ (alebo teda signálu $\dot f(t)$), teda
\begin{align}
    \mathcal L \left\{ \frac{\text{d}f(t)}{dt} \right\} = \int_0^\infty \frac{\text{d}f(t)}{\text{d}t} e^{-st}\text{d}t
\end{align}
Tento integrál je možné nájsť metódou per partes, pri ktorej vo všeobecnosti platí
\begin{align}
    \int_0^\infty u(t)v^\prime(t)\text{d}t = \left[ u(t)v(t) \right]_0^\infty - \int_0^\infty u^\prime(t) v(t) \text{d}t
\end{align}
Uvažujme tu $u(t) = e^{-st}$ a $v(t) = f(t)$, potom
\begin{equation}
    \begin{aligned}
        \int_0^\infty \frac{\text{d}f(t)}{\text{d}t} e^{-st}\text{d}t
            &=  \left[ e^{-st} f(t) \right]_0^\infty - (-s)  \int_0^\infty f(t) e^{-st}\text{d}t \\
            &= 0 - f(0) + s F(s) \\
            &= s F(s) - f(0)
    \end{aligned}
\end{equation}
je L-obraz signálu $\frac{\text{d}f(t)}{dt}$.






\subsection{Integrál}

Obdobne by sme mohli hľadať aj obraz signálu $\int_0^t f(\tau) \text{d}\tau$, teda
\begin{align}
    \mathcal L \left\{ \int_0^t f(\tau) \text{d}\tau \right\} = \int_0^\infty \left(\int_0^t f(\tau) \text{d}\tau \right) e^{-st}\text{d}t
\end{align}
Hľadajme L-obraz tak, že zavedieme signál $g(t) = \int_0^t f(\tau) \text{d}\tau$ čo potom znamená, že $\dot g(t) = f(t)$. Hľadáme $\mathcal L \left\{ g(t) \right\} = G(s)$. Najskôr si však všimnime, že
\begin{equation}
    \begin{aligned}
        \mathcal L \left\{ \dot g(t) \right\} =& s G(s) - g(0) \\
        & s G(s) - g(0) = F(s)
    \end{aligned}
\end{equation}
a k tomu vidíme, že $g(0) = \int_0^0 f(\tau) \text{d}\tau = 0$. Teda
\begin{subequations}
    \begin{align}
        sG(s) &= F(s) \\
        G(s) &= \frac{1}{s} F(s)
    \end{align}
\end{subequations}
čím sme našli
\begin{align}
    \mathcal L \left\{ \int_0^t f(\tau) \text{d}\tau \right\} = \frac{1}{s} F(s)
\end{align}






\subsection{Obraz Dirackovho impulzu}

Dirackov impulz je signál taký, že (napríklad)
\begin{align}
    \delta(t) =
    \left\{
        \begin{aligned}
            &0 & \text{ak $t \neq 0$} \\
            &\infty & \text{ak $t = 0$}
        \end{aligned}
    \right.
\end{align}
pričom z princípu platí
\begin{align}
    \int_{-\infty}^\infty \delta(\tau) \text{d}\tau = 1
\end{align}

Totiž, v závislosti od toho ako by sme presnejšie matematicky špecifikovali Dirackov impulz $\delta(t)$ by sa konkrétne spôsoby aplikácie LT (výpočet integrálu) mohli formálne líšiť, avšak v každom prípade vždy platí
\begin{align}
    \mathcal L \left\{ \delta(t) \right\} = 1
\end{align}





\subsection{Obraz jednotkového skoku}
Pri tzv. jednotkovom skoku sa uvažuje, že v čase $0$ sa hodnota signálu skokovo zmení z $0$ na $1$ (má hodnotu „jedna jednotka“). Keďže sa tu nachádzame len v čase väčšom ako nula, môžeme uvažovať, že tu hľadáme obraz signálu $f(t) = 1$, teda
\begin{equation}
    \begin{aligned}
        \mathcal L \left\{ 1 \right\} &= \int_0^\infty 1 e^{-st}\text{d}t \\
        &= \left[ - \frac{1}{s} e^{-st} \right]_0^\infty \\
        &= 0 - \frac{1}{s} e^{-s0} \\
        &= \frac{1}{s}
    \end{aligned}
\end{equation}



\subsection{Obraz exponencialnej funkcie}
\label{vyhlcast}

Nájdime obraz $f(t) = e^{at}$.
\begin{equation}
    \begin{aligned}
        F(s) &= \int_0^\infty e^{at} e^{-st}\text{d}t \\
        &= \int_0^\infty e^{(a-s)t}\text{d}t \\
        &= \left[ \frac{1}{a-s} e^{(a-s)t} \right]_0^\infty \\
        &= 0 - \frac{1}{a-s} \\
        &= \frac{1}{s - a}
    \end{aligned}
\end{equation}




\subsection{Obraz časového posunutia}

Majme signál $f(t)$. Signál posunutý v čase je $f(t-D)$ (v zmysle vstupno-výstupného oneskorenia, alebo dopravného oneskorenia). Obrazom $f(t)$ je $F(s)$. Obrazom $f(t-D)$ je
\begin{align}
    \int_0^\infty f(t-D) e^{-st} \text{d}t
\end{align}
Zaveďme substitúciu $\tau  = t-D$, teda $t=\tau+D$ a tiež $\text{d}t = \text{d}\tau$ keďže $D$ je v čase konštantné. Potom
\begin{align}
    \int_0^\infty f(\tau) e^{-s(\tau+D)} \text{d}\tau = e^{-sD} \int_0^\infty f(\tau) e^{-s\tau} \text{d}\tau
\end{align}
a je zrejmé, že
\begin{align}
    e^{-sD} F(s)
\end{align}
je obrazom posunutého signálu $f(t-D)$.




\section{Inverzná Laplaceova transformácia}

Na tomto mieste je vhodné uviesť opak Laplaceovej transformácie, teda inverznú Laplaceovu transformáciu. Značíme ju ako
\begin{align}
    \mathcal L ^{-1} \left\{ F(s) \right\} = f(t)
\end{align}
pričom formálne ide o operáciu definovanú vzťahom
\begin{align}
    f(t) = \frac{1}{2\pi j} \int_{\sigma-j\omega}^{\sigma + j\omega} F(s) e^{st} \text{d}s
\end{align}

Výpočet inverznej LT spravidla nie je jednoduchý. V praxi sa využíva tabuľka Laplaceových obrazov signálov, ktorá uvádza L-obrazy a k nim prislúchajúce časové signály. Tabuľka obsahuje výber typických a dôležitých signálov využívaných pri analýze dynamických systémov.

Zložitý obraz riešenia diferenciálnej rovnice je zväčša možné upraviť tak, že je v~ňom vidieť jednotlivé dielčie obrazy zodpovedajúce typickým signálom (uvedeným v tabuľke). Z typických časových signálov sa potom vyskladá časová funkcia zodpovedajúca celkovému riešeniu (v časovej oblasti).






\section{Tabuľka Laplaceových obrazov signálov}




\newcommand{\Laplace}[1]{\ensuremath{\mathcal{L}{\left\{#1\right\}}}}
\newcommand{\InvLap}[1]{\ensuremath{\mathcal{L}^{-1}{\left\{#1\right\}}}}


\noindent
\begin{longtable}[l]{p{3.7cm} @{} p{5.9cm} p{2.8cm}}

    \toprule
    $f(t)$                                  & $\Laplace{f(t)}$   & {\color{Gray} \scriptsize Poznámka} \\
    \midrule
    \addlinespace[5mm]
    \endhead



    $f(t)$                                  & $F(s)$                                \\[4mm]
    $\dot f(t)$                             & $sF(s) - f(0)$                        \\[4mm]
    $\displaystyle \frac{\text{d}^n f(t)}{\text{d}t^n}$                                 & $s^nF(s) - s^{(n-1)} f(0) - \cdots - f^{(n-1)}(0)$ \\[4mm]    
    \midrule \addlinespace[4mm]


    $1$                                     & $\dfrac{1}{s}$                        & {\color{Gray} \scriptsize Skoková zmena v~čase~$0$}  \\[4mm]
    $t^n$ ($n=0,1,2,\dots$)                 & $\dfrac{n!}{s^{n+1}}$                 \\[4mm]
    \midrule \addlinespace[4mm]


    $\delta(t)$                             & $1$                                   & {\color{Gray} \scriptsize Dirackov impulz} \\[4mm]
    $\delta(t-t_0)$                         & $1 \, e^{-st_0}$                      & {\color{Gray} \scriptsize Časové oneskorenie} \\[4mm]
    \midrule \addlinespace[4mm]
        
    
    $e^{at}$                                & $\dfrac{1}{s-a}$                      \\[4mm]
    $e^{-at}$                                & $\dfrac{1}{s+a}$                     \\[4mm]
    \midrule \addlinespace[4mm]


    $\sin(kt)$                               & $\dfrac{k}{s^2+k^2}$                  \\[4mm]
    $\cos(kt)$                               & $\dfrac{s}{s^2+k^2}$                  \\[4mm]
    $\sinh(kt)$                              & $\dfrac{k}{s^2-k^2}$                  \\[4mm]
    $\cosh(kt)$                              & $\dfrac{s}{s^2-k^2}$                  \\[4mm]
    \midrule \addlinespace[4mm]


    $\displaystyle{\int_0^t f(x)g(t-x) \text{d}x}$ & $F(s)G(s)$                     & {\color{Gray} \scriptsize Konvolučný integrál} \\[4mm]
    \midrule \addlinespace[4mm]


    $t^nf(t)$                               & $(-1)^n\dfrac{\text{d}^n F(s)}{\text{d} s^n}$ \\[4mm]
    $te^{at}$                               & $\dfrac{1}{(s-a)^2}$                  \\[4mm]
    $t^ne^{at}$                             & $\dfrac{n!}{(s-a)^{n+1}}$             \\[4mm]
    $t\sin kt$                              & $\dfrac{2ks}{(s^2+k^2)^2}$            \\[4mm]
    $t\cos kt$                              & $\dfrac{s^2-k^2}{(s^2+k^2)^2}$        \\[4mm]
    $t\sinh kt$                             & $\dfrac{2ks}{(s^2-k^2)^2}$            \\[4mm]
    $t\cosh kt$                             & $\dfrac{s^2+k^2}{(s^2-k^2)^2}$        \\[4mm]
    \midrule \addlinespace[4mm]

    
    $e^{at}f(t)$                            & $F(s-a)$                              \\[4mm]
    $e^{at}\sin kt$                         & $\dfrac{k}{(s-a)^2+k^2}$              \\[4mm]
    $e^{at}\cos kt$                         & $\dfrac{s-a}{(s-a)^2+k^2}$            \\[4mm]
    $e^{at}\sinh kt$                        & $\dfrac{k}{(s-a)^2-k^2}$              \\[4mm]
    $e^{at}\cosh kt$                        & $\dfrac{s-a}{(s-a)^2-k^2}$            \\[4mm]
    % \midrule \addlinespace[4mm]


    % $\dfrac{e^{at}-e^{bt}}{a-b}$            & $\dfrac{1}{(s-a)(s-b)}$               \\[4mm]
    % $\dfrac{ae^{at}-be^{bt}}{a-b}$          & $\dfrac{s}{(s-a)(s-b)}$               \\[4mm]
    % $\dfrac{\sin at}{t}$                    & $\arctan \dfrac{a}{s}$                \\[4mm]
    % $\dfrac{1}{\sqrt{\pi t}}e^{-a^2/4t}$    & $\dfrac{e^{-a\sqrt{s}}}{\sqrt{s}}$    \\[4mm]
    % $\dfrac{a}{2\sqrt{\pi t^3}}e^{-a^2/4t}$ & $e^{-a\sqrt{s}}$                      \\[4mm]

    \bottomrule

\end{longtable}






\section{Laplaceov obraz a originál riešenia diferenciálnej rovnice}

\subsection{Príklad s homogénnou diferenciálnou rovnicou}

Majme diferenciálnu rovnicu
\begin{align}  \label{prrov}
    \dot y(t) - a y(t) = 0 \qquad y(0) = y_0
\end{align}
Na jednotlivé signály v tejto rovnici aplikujme LT.
\begin{align}
    \left( s Y(s) - y(0)  \right) - a Y(s) = 0
\end{align}
kde $Y(s)$ je obrazom signálu $y(t)$. $Y(s)$ je teda obrazom riešenia rovnice. Vyjadrime $Y(s)$:
\begin{align}
    \begin{aligned}
        (s-a)Y(s) - y(0) &= 0 \\
        Y(s) &= \frac{1}{(s-a)} y(0)
    \end{aligned}
\end{align}

Otázka je, ak poznáme signál v s-oblasti (v Laplaceovej doméne), vieme určiť pôvodný signál v časovej oblasti? Vieme nájsť pomocou obrazu riešenia $Y(s)$ samotné riešenie $y(t)$?

V tomto prípade je vzhľadom na časť~\ref{vyhlcast} jasné, že
\begin{align} \label{prries}
    \mathcal L ^{-1} \left\{ Y(s) \right\} = y(t) = e^{at}y(0)
\end{align}
kde $\mathcal L ^{-1} \left\{  \right\}$ predstavuje  inverznú LT transformáciu. Tiež je jasné, že \eqref{prries} je správne riešenie diferenciálnej rovnice \eqref{prrov}.






\subsection{Príklad s nehomogénnou diferenciálnou rovnicou}


Majme rovnicu
\begin{align}
    \ddot y(t) +4 \dot y(t) + 3y(t) = u(t) \qquad y(0) = 3, \dot y(0) = -2
\end{align}
kde vstupný signál $u(t) = 12$ (konštantný v čase). Aplikujme LT
\begin{subequations}
\begin{align}
    \big( s \mathcal L \{\dot y\} - \dot y(0) \big) + 4 \left( sY(s) - y(0) \right) + 3 Y(s) &=  U(s) \\
    \Big( s \big( sY(s) - y(0) \big) - \dot y(0) \Big) + 4sY(s) - 4y(0) + 3Y(s) &= U(s) \\
    s^2Y(s) - sy(0) - \dot y(0)  + 4sY(s) - 4y(0) + 3Y(s) &= U(s) \\
    s^2Y(s)   + 4sY(s)  + 3Y(s) - sy(0) - \dot y(0) - 4y(0) &= U(s)
\end{align}
\end{subequations}
a teda
\begin{subequations}
\begin{align}
    \left( s^2   + 4s  + 3\right)Y(s) &= sy(0) +  \dot y(0) + 4y(0) + U(s) \\
    Y(s) &= \frac{sy(0) +  \dot y(0) + 4y(0)}{\left( s^2   + 4s  + 3\right)} + \frac{1}{\left( s^2   + 4s  + 3\right)}U(s)
\end{align}
\end{subequations}

\vbox{
Poznáme aj konkrétny tvar obrazu $U(s)$, keďže $u(t) = 12$, tak $U(s) = 12 \frac{1}{s}$, teda
\begin{align}
    Y(s) &= \frac{sy(0) +  \dot y(0) + 4y(0)}{\left( s^2   + 4s  + 3\right)} + \frac{1}{\left( s^2   + 4s  + 3\right)} 12 \frac{1}{s}
\end{align}
a toto je obrazom riešenia diferenciálnej rovnice.
}

Všimnime si, že sú tu prítomné dve zložky
\begin{align} \label{vvzlozky}
    Y(s)
    &=
    \underbrace{
    \frac{3s + 10}{\left( s^2   + 4s  + 3\right)}
    }_{\text{vlastná zložka}}
    +
    \underbrace{
    \frac{12}{\left( s^2   + 4s  + 3\right) s}
    }_{\text{vnútená zložka}}
\end{align}
kde sme aj číselne dosadili hodnoty začiatočných podmienok.


Keď je obraz riešenia v tvare \eqref{vvzlozky} je prakticky nemožné priradiť k nemu originálny časový signál -- nie sú tam očividné typické obrazy typických signálov.

Rozložme na parciálne zlomky
\begin{align}
    \frac{3s + 10}{\left( s^2   + 4s  + 3\right)} &= \frac{7}{2(s+1)} - \frac{1}{2(s+3)} \label{pz1} \\
    \frac{12}{\left( s^2   + 4s  + 3\right) s} &= \frac{4}{s} - \frac{6}{(s+1)} + \frac{2}{(s+3)} \label{pz2}
\end{align}
a tým sa hneď stáva zrejmé, že \eqref{pz1} má originál
\begin{align}
    y_{vlast}(t) = \frac{7}{2} e^{-t} - \frac{1}{2} e^{-3t}
\end{align}
a \eqref{pz2} ma originál
\begin{align}
    y_{vnut}(t) = 4 - 6 e^{-t} + 2 e^{-3t}
\end{align}
Celkové riešenie je
\begin{align}
    \begin{aligned}
        y(t) &= \frac{7}{2} e^{-t} - \frac{1}{2} e^{-3t} +  4 - 6 e^{-t} + 2 e^{-3t} \\
        &= 4 - \frac{5}{2} e^{-t} + \frac{3}{2} e^{-3t}
    \end{aligned}
\end{align}
















\section[Súvislosti so všeobecným riešením nehomogénnych diferenciálnych rovníc]{Súvislosti so všeobecným riešením\\ nehomogénnych diferenciálnych rovníc}
\label{predchcasttato}

Obdobne ako pri hľadaní riešenia homogénnej dif. rovnice, kde sa ako východisko predpokladá riešenie v tvare exponenciálnej funkcie $e^{s t}$, tak pri hľadaní riešenia nehomogénnej dif. rovnice je možné skúmať predpoklad, že vstupný signál je v tvare exponenciálnej funkcie $e^{s t}$. 




Najskôr pripomeňme, že riešením homogénnej dif. rovnice 
\begin{equation}
    \dot y(t) + a y(t) = 0 \qquad y(0) = y_0
\end{equation}
je
\begin{equation}
    y(t) =  e^{-a t} y_0
\end{equation}
a~ide tu o rovnicu prvého rádu. 

Formálne je však aj tu možné uplatniť rozklad dif. rovnice vyššieho rádu na sústavu rovníc prvého rádu v zmysle
\begin{subequations}
    \begin{align}
        \dot x(t) &= a x(t) \qquad x(0) = x_0 \\
        y(t) &= x(t)
    \end{align}    
\end{subequations}
kde $a \in \mathbb{R}$ a $x(t)$ je stavová veličina. Pri dif. rovnici vyššieho rádu by $x(t)$ bol vektor stavových veličín a~udával by sústavu rovníc v tvare
\begin{subequations}
    \begin{align}
        \dot x(t) &= A x(t) \qquad x(0) = x_0  \\
        y(t) &= c^\naT x(t)
    \end{align}    
\end{subequations}
kde $A \in \mathbb{R}^{n\times n} $ je matica, $c \in \mathbb{R}^{n}$ je vektor a~$x_0 \in \mathbb{R}^{n}$ je vektor. Riešením je
\begin{equation}
    y(t) = c^\naT e^{A t} x_0
\end{equation}
kde sme využili objekt $e^{At}$ čo je tzv. maticová exponenciálna funkcia. Tu sa jej definícii nebudeme venovať podrobne, čitateľa odkazujeme napr. na \cite{Aastroem2020}. Ide zjavne o zovšeobecnenie skalárneho prípadu (systémy prvého rádu) pre vektorový prípad (systémy vyššieho rádu). Definícia a~následné využívanie matice $e^{At}$ je základom pre pojmy ako fundamentálne riešenia systému (diferenciálnej rovnice). Samotná matica $e^{At}$ sa označuje napríklad aj ako matica fundamentálnych riešení. Takpovediac „účinok“ matice $e^{At}$ je daný maticou $A$, a~tú možno charakterizovať jej vlastnými číslami (a~vlastnými vektormi). Tieto sú následne zdrojom definície pojmu charakteristická rovnica tak ako sa to využíva pri hľadaní analytického riešenia diferenciálnej rovnice.

V prípade nehomogénnej dif. rovnice je systém daný sústavou rovníc v tvare
\begin{subequations} \label{sysLTIstavpries}
    \begin{align}
        \dot x(t) &= A x(t) + b u(t) \qquad x(0) = x_0 \\
        y(t) &= c^\naT x(t)
    \end{align}
\end{subequations}
kde $u(t)$ je vstupný signál, $b \in \mathbb{R}^{n}$ je vektor. Je možné ukázať, že
\begin{align} \label{xriesLTI}
	x(t) = e^{At} x(0) + \int_0^t e^{A(t-\tau)} b u(\tau) \text{d}\tau
\end{align}
a teda samotné riešenie (výstupný signál $y(t)$) je
\begin{subequations}
    \begin{align}
        y(t) &= c^\naT x(t) \\
        y(t) &= c^\naT e^{At} x(0) + \int_0^t c^\naT e^{A(t-\tau)} b u(\tau) \text{d}\tau \label{riesnhrov}
    \end{align}
\end{subequations}
Prvý člen (na pravej strane rovnice \eqref{riesnhrov}) sa nazýva \emph{vlastná zložka riešenia} (je vyvolaná začiatočnými podmienkami) a druhý člen sa nazýva  \emph{vnútená zložka riešenia} (je vyvolaná vstupným signálom).

Ako sme uviedli, zámerom je skúmať predpoklad, že vstupný signál je v tvare exponenciálnej funkcie
\begin{align}
	u(t) = e^{st}
\end{align}
kde $s = \sigma + j\omega$ (vo všeobecnosti). To, že $s$ je komplexné číslo (komplexná premenná) umožňuje považovať tento špeciálny signál vlastne za triedu signálov (rôzneho typu). Reálna časť premennej $s$ určuje exponenciálny rast alebo pokles (dokonca ak $s = 0$ potom je špeciálny signál vlastne konštantným) a imaginárna časť určuje harmonické kmitanie signálu.

Máme \eqref{xriesLTI}, a teda:
\begin{align}
	x(t) = e^{At} x(0) + \int_0^t \left( e^{A(t-\tau)} b e^{s\tau} \right) \text{d}\tau
\end{align}
kde pri manipulácii s výrazom $ \left( e^{A(t-\tau)} b e^{s\tau} \right)$ treba manipulovať s ohľadom na fakt, že ide o matice a vektory. V každom prípade, po integrácii sa získa
\begin{align}
	x(t) = e^{At} x(0) + e^{At} \left( sI - A \right)^{-1}  \left( e^{(sI-A)t} - I \right) b
\end{align}
kde $I$ je jednotková matica.

Celkové riešenie, inými slovami výstupný signál systému, potom je
\begin{align}
	\begin{aligned}
		y(t) &= c^\naT e^{At} x(0) + c^\naT e^{At} \left( sI - A \right)^{-1}  \left( e^{(sI-A)t} - I \right) b \\
		&= c^\naT e^{At} x(0) + c^\naT e^{At} \left( sI - A \right)^{-1}  \left( e^{st} e^{-At} - I \right) b \\
		&= c^\naT e^{At} x(0) + c^\naT e^{At} \left( sI - A \right)^{-1}  \left( e^{st} e^{-At}b - b \right) \\
		&= c^\naT e^{At} x(0) +   \left(c^\naT e^{At} \left( sI - A \right)^{-1} e^{st} e^{-At}b - c^\naT e^{At} \left( sI - A \right)^{-1} b \right) \\
		&= c^\naT e^{At} x(0) +   \left(c^\naT \left( sI - A \right)^{-1} e^{st} b - c^\naT e^{At} \left( sI - A \right)^{-1} b \right)
	\end{aligned}
\end{align}
V tomto bode je možné konštatovať:
\begin{align}
	\begin{aligned}
		y(t)
		&=
		\underbrace{
		c^\naT e^{At} x(0)
		}_{\text{vlastná zložka}}
		+
		\underbrace{
		\left(c^\naT \left( sI - A \right)^{-1} e^{st} b - c^\naT e^{At} \left( sI - A \right)^{-1} b \right)
		}_{\text{vnútená zložka}}
	\end{aligned}
\end{align}
a zároveň:
\begin{align}
    \begin{split}
        y(t) 
        &= 
        c^\naT e^{At} \left( x(0) -  \left( sI - A \right)^{-1} b\right) +   \left(c^\naT  \left( sI - A \right)^{-1}  b e^{st} \right) 
        \\
        &=
        \underbrace{
        c^\naT e^{At} \left( x(0) -  \left( sI - A \right)^{-1} b\right)
        }_{\text{zložka opisujúca prechodné deje}}
        +
        \underbrace{
        \left(c^\naT  \left( sI - A \right)^{-1}  b  \right) e^{st}
        }_{\text{čisto exponenciálna zložka}}
    \end{split}
\end{align}

O vplyve samotného špeciálneho signálu $e^{st}$ na celkové riešenie teda rozhoduje výraz $c^\naT  \left( sI - A \right)^{-1}  b$. Formálne sa
\begin{equation}
	G(s) = c^\naT  \left( sI - A \right)^{-1}  b
\end{equation}
nazýva prenosová funkcia systému.

Uvedené je založené na fakte vyjadrenom všeobecným riešením \eqref{xriesLTI} pričom ide o~riešenie sústavy dif. rovníc prvého rádu v tvare \eqref{sysLTIstavpries}. Takpovediac pôvodná dif. rovnica vyššieho rádu je pre tento prípad v tvare
\begin{equation} \label{vseobDifRov_nh2}
	\frac{\text{d}^n y(t)}{\text{d}t^n} + a_{n-1} \frac{\text{d}^{(n-1)} y(t)}{\text{d}t^{(n-1)}} + \cdots + a_0 y(t) = b_m \frac{\text{d}^m u(t)}{\text{d}t^m} + b_{m-1} \frac{\text{d}^{m-1} u(t)}{\text{d}t^{m-1}} + \cdots + b_0 u(t)
\end{equation}
Potom ak na vstupe uvažujeme $u(t) = e^{st}$ a zároveň vieme, že riešenie systému je tiež nejaký exponenciálny signál, čo možno vo všeobecnosti vyjadriť ako $y(t) = y_0 e^{st}$ (kde $y_0$ najmä odlišuje $y(t)$ od $u(t)$). Ak $y(t)$ a $u(t)$ dosadíme do \eqref{vseobDifRov_nh2}, vidíme, že
\begin{equation}
    \begin{aligned}
                \frac{\text{d}^n y_0 e^{st}}{\text{d}t^n}
        + a_{n-1} \frac{\text{d}^{(n-1)} y_0 e^{st}}{\text{d}t^{(n-1)}}
        + \cdots
        + a_0 y_0 e^{st}
        &=
        b_m \frac{\text{d}^m e^{st}}{\text{d}t^m}
        + b_{m-1} \frac{\text{d}^{m-1} e^{st}}{\text{d}t^{m-1}}
        + \cdots
        + b_0 e^{st}
        \\
        y_0 e^{st} s^n
        + a_{n-1}  y_0 e^{st} s^{(n-1)}
        + \cdots
        + a_0 y_0 e^{st}
        &=
        b_m  e^{st} s^m
        + b_{m-1}  e^{st} s^{m-1}
        + \cdots
        + b_0 e^{st}
        \\
        \left(
            s^n
            + a_{n-1}   s^{(n-1)}
            + \cdots
            + a_0
        \right)
        y_0 e^{st}
        &=
        \left(
        b_m   s^m
        + b_{m-1}   s^{m-1}
        + \cdots
        + b_0
        \right)
        e^{st}
        \\
        y_0 e^{st}
        &=
        \frac{
        \left(
        b_m   s^m
        + b_{m-1}   s^{m-1}
        + \cdots
        + b_0
        \right)
        }{
        \left(
            s^n
            + a_{n-1}   s^{(n-1)}
            + \cdots
            + a_0
        \right)
        }
        e^{st}
    \end{aligned}
\end{equation}
a teda môžeme povedať, že riešenie systému závislé od špeciálneho signálu $e^{st}$ je
\begin{equation}
    y(t)
    =
    \frac{
    \left(
    b_m   s^m
    + b_{m-1}   s^{m-1}
    + \cdots
    + b_0
    \right)
    }{
    \left(
    s^n
    + a_{n-1}   s^{(n-1)}
    + \cdots
    + a_0
    \right)
    }
    e^{st}
\end{equation}

Označme
\begin{subequations}
	\begin{align}
		B(s) &= \left( b_m   s^m + b_{m-1}   s^{m-1} + \cdots + b_0 \right) \\
		A(s) &=  \left( s^n + a_{n-1}   s^{(n-1)} + \cdots + a_0 \right)
	\end{align}
\end{subequations}
a výraz
\begin{equation}
	G(s) = \frac{B(s)}{A(s)}
\end{equation}
vyjadruje prenosovú funkciu systému.







\section{O prenosovej funkcii}


Prenosová funkcia je nástroj pre matematické modelovanie lineárnych časovo-invariantných dynamických systémov.

Primárne sú dynamické systémy opisované diferenciálnymi rovnicami. Ak sú tieto rovnice lineárne, hovoríme, že systém, ktorý opisujú, je lineárny. Ak koeficienty v dif. rovnici nie sú funkciami času, hovoríme, že systém je časovo invariantný.

Vo všeobecnosti hľadáme riešenie dif. rovnice. V kontexte dynamických systémov je riešením dif. rovnice funkcia času. Z hľadiska systému hovoríme, že táto funkcia je výstupným signálom systému. Na hľadané riešenie má vplyv niekoľko faktorov. Vo všeobecnosti je riešenie dané samozrejme samotnou dif. rovnicou, jej rádom a~hodnotami jej koeficientov. Konkrétne riešenia sú potom dané začiatočnými podmienkami a~vstupným signálom systému.

Z hľadiska systému hovoríme o ráde dif. rovnice a~o jej koeficientoch ako o~parametroch systému. Hovoriť o začiatočných podmienkach systému má samozrejme tiež význam. Napokon nás zaujíma vplyv vstupného signálu na výstupný signál systému a~s matematickým modelovaním tohto vplyvu súvisí pojem prenosová funkcia. Obrazne hovoríme o prenose zo vstupu na výstup systému.





\subsection{Definícia prenosovej funkcie s využitím Laplaceovej transformácie}

Prenosová funkcia je definovaná ako pomer Laplaceovho obrazu výstupného signálu systému k Laplaceovmu obrazu vstupného signálu pri nulových začiatočných podmienkach systému.

Laplaceova transformácia sa týka lineárnych časovo invariantných systémov. Majme teda takýto systém, ktorého vstupným signálom je signál $u(t)$ a výstupným signálom je signál $y(t)$. V zmysle Laplaceovej transformácie existuje obraz vstupného signálu $U(s)$ a obraz výstupného signálu $Y(s)$ pričom tieto obrazy sú stanovené pri nulových začiatočných podmienkach systému.

Ilustrujme na príklade. Lineárny časovo invariantný systém nech je daný dif. rovnicou v tvare
\begin{equation}
    a_1 \dot y(t) + a_0 y(t) = b_0 u(t)
\end{equation}
kde $y(t)$ a $u(t)$ sú samozrejme výstupný a vstupný signál. Koeficienty $a_1, a_0, b_0 \in \mathbb R$ sú konštantné. Aplikujme Laplaceovu transformáciu na prvky danej dif. rovnice.
\begin{align}
    \begin{aligned}
        a_1 \mathcal{L} \left\{ \dot y(t) \right\} + a_0 \mathcal{L} \left\{ y(t) \right\} &= b_0 \mathcal{L} \left\{ u(t) \right\} \\
        a_1 s Y(s) - a_1 y(0) + a_0 Y(s) &= b_0 U(s)
    \end{aligned}
\end{align}
Pri nulových začiatočných podmienkach potom platí
\begin{equation}
    a_1 s Y(s) + a_0 Y(s) = b_0 U(s)
\end{equation}
Prenosová funkcia je definovaná ako pomer $Y(s)/U(s)$, teda
\begin{subequations}
    \begin{align}
        Y(s) \left( a_1 s + a_0 \right) &= b_0 U(s) \\
        \frac{Y(s)}{U(s)} &= \frac{b_0}{a_1 s + a_0}
    \end{align}
\end{subequations}
Ako samostatný objekt sa prenosová funkcia označuje samostatne, napríklad sko $G(s)$, teda v tomto prípade
\begin{equation} \label{prenosfunkciapriklad}
    G(s) = \frac{b_0}{a_1 s + a_0}  
\end{equation}
a vo všeobecnosti
\begin{equation}
    G(s) = \frac{Y(s)}{U(s)}
\end{equation}
Z iného hľadiska má tiež zmysel samostatne označovať polynómy v čitateli a menovateli prenosovej funkcie. Polynóm v čitateli sa typicky označuje ako $B(s)$ a polynóm v~menovateli sa označuje ako $A(s)$. V tomto prípade 
\begin{equation}
    B(s) = b_0 \qquad A(s) = a_1 s + a_0
\end{equation}
a vo všeobecnosti teda
\begin{equation}
    G(s) = \frac{B(s)}{A(s)}
\end{equation}

    





\subsection{Súvisiace pojmy}

Prenosová funkcia je opisuje lineárny časovo invariantný dynamický systém pričom je dané, že začiatočné podmienky systému sú nulové. Daný dynamický systém je možné opísať diferenciálnou rovnicou alebo prenosovou funkciou a tieto dva opisy sú ekvivalentné. 

Uvažujme prenosovú funkciu vo všeobecnosti
\begin{equation}
    G(s) = \frac{B(s)}{A(s)}
\end{equation}
kde $A(s)$ a $B(s)$ sú polynómy, ktorých nezávisle premenná je Laplaceov operátor $s$~(pritom $s$ je komplexné číslo). 

Polynóm $A(s)$ má stupeň $n$ a polynóm $B(s)$ má stupeň $m$.

Reálne/skutočné dynamické deje/systémy „v prírode“ sú samozrejme kauzálne\footnote{Nekauzalita je skôr matematická/abstraktná záležitosť.}, teda výstup je následkom diania v súčastnosti a minulosti. Ak prenosová funkcia opisuje kauzálny systém, potom pre stupne polynómov $A(s)$ a $B(s)$ platí $n \geq m$.

Polynóm $A(s)$ sa nazýva charakteristický polynóm prenosovej funkcie. Pojem charakteristická rovnica alebo charakteristický polynóm je používaný aj v kontexte analytických metód riešenia lineárnych dif. rovníc. Ide pritom o ekvivalentné pojmy, charakteristický polynóm prenosovej funkcie je to isté ako charakteristický polynóm lineárnej diferenciálnej rovnice.

Stupeň polynómu $A(s)$, teda hodnota $n$, sa nazýva rád prenosovej funkcie. Ide o~ekvivalent pojmu rád dynamického systému (najvyšší stupeň derivácie neznámej v~dif. rovnici).

Korene polynómu $A(s)$ sa nazývajú póly prenosovej funkcie. Ekvivalentne je možné hovoriť o póloch lineárneho dynamického systému. Keďže ide o korene charakteristického polynómu, s pólmi systému priamo súvisia fundamentálne riešenia dif. rovnice. Fundamentálne riešenia sú dané pólmi systému. Iný termín pre fundamentálne riešenia je \emph{módy dynamického systému}.

Z hľadiska stability dynamického systému hovoríme, že systém je stabilný ak sú všetky póly systému v ľavej polrovine komplexnej roviny. Inými slovami, systém je stabilný ak reálne časti všetkých pólov sú záporné. Pod stabilitou systému samozrejme myslíme stabilitu rovnovážneho stavu daného lineárneho dynamického systému.


Korene polynómu $B(s)$ sa nazývajú nuly prenosovej funkcie (nuly lineárneho dynamického systému).

Nuly systému súvisia predovšetkým so vstupným signálom systému. Širšia interpretácia prenosovej funkcie, ako vieme, sa zaoberá skúmaním vplyvu exponenciálneho vstupného signálu $u(t) = e^{st}$ ($s$ je komplexné číslo) na výstup systému. Zjednodušene povedané, nuly nulujú zodpovedajúce vstupné exponenciálne signály. Neprenesú sa na výstup. Poloha nuly v komplexnej rovine určuje signál $e^{st}$, ktorý je nulovaný a~neprenesie sa na výstup (neovplyvní výstupnú veličinu). Ďalšia diskusia v tejto veci je nad rámec tohto textu a čitateľ sa odkazuje na zodpovedajúcu literatúru, napríklad aj \cite{Aastroem2020}.

V súvislosti s prenosovou funkciou, o ktorej uvažujeme v tvare
\begin{equation}
    G(s) = \frac{Y(s)}{U(s)}
\end{equation}
je užitočné využívať \emph{vetu o konečnej hodnote riešenia}. Ak máme k dispozícii obraz riešenia diferenciálnej rovnice, teda obraz výstupného signálu systému $Y(s)$, potom veta o konečnej hodnote hovorí, že konečná hodnota výstupného signálu $y(t)$, označme túto hodnotu symbolom $y(\infty)$, je daná ako
\begin{equation}
    y(\infty) = \lim_{s \to 0} s\ Y(s)   
\end{equation}
Napríklad, poznáme prenosovú funkciu \eqref{prenosfunkciapriklad} a napríklad vstupom systému je jednotkový skok, ktorého Laplaceov obraz je $U(s) = 1/s$. Potom obraz výstupného signálu je
\begin{equation}
    Y(s) = G(s) U(s) = \left( \frac{b_0}{a_1 s + a_0} \right) \frac{1}{s}  
\end{equation}
Konečná hodnota tohto signálu bude
\begin{subequations}
    \begin{align}
        y(\infty) &= \lim_{s \to 0} s\ Y(s) = \lim_{s \to 0} s \left( \frac{b_0}{a_1 s + a_0} \right) \frac{1}{s} 
        \\
        &= \lim_{s \to 0} \left( \frac{b_0}{a_1 s + a_0} \right) = \frac{b_0}{a_0}        
    \end{align}
\end{subequations}





\section{Algebra prenosových funkcií}

Prenosová funkcia je nástroj pre matematické modelovanie lineárnych časovo-invariantných dynamických systémov. Prenosovú funkciu je možné vidieť aj ako jeden blok v blokovej schéme, teda:

\begin{center}

    \makebox[\textwidth][c]{%
    \input{../fig_standalone/TFalgebra_lenG.pdf_tex}
    }

	\figcaption{Prenosová funkcia ako jeden blok v blokovej schéme}
	\label{TFalgebra_lenG}

\end{center}

Manipulácia s takýmito blokmi je jednou z aplikácií algebry prenosových funkcií. V tomto zmysle je potrebné uvažovať tri základné situácie. Sériové zapojenie blokov, paralelné zapojenie blokov a spätnoväzbové zapojenie blokov.




\subsection{Sériové zapojenie blokov}

Uvažujme systém, ktorý je tvorený kaskádnou kombináciou dvoch podsystémov. Prenosové funkcie podsystémov sú $G_1(s)$ a $G_2(s)$. Vstup prvého podsystému je zároveň vstupom celkového systému. Výstup prvého podsystému je vstupom druhého podsystému. Výstup druhého podsystému je zároveň výstupom celkového systému. Ide o sériové zapojenie podsystémov.

\begin{center}

    \makebox[\textwidth][c]{%
    \input{../fig_standalone/TFalgebra_seriove.pdf_tex}
    }

	\figcaption{Sériové zapojenie blokov}
	\label{TFalgebra_seriove}

\end{center}

Hľadáme prenosovú funkciu celkového systému, označme ju $G(s)$. Pre sériové zapojenie podsystémov platí
\begin{align}
    G(s) = G_1(s)\ G_2(s)
\end{align}
Výslednú prenosovú funkciu teda získame súčinom prenosových funkcií podsystémov.




\subsection{Paralelné zapojenie blokov}

\begin{center}

    \makebox[\textwidth][c]{%
    \input{../fig_standalone/TFalgebra_paralelne.pdf_tex}
    }

	\figcaption{Paralelné zapojenie blokov}
	\label{TFalgebra_paralelne}

\end{center}

Pri paralelnom zapojení podsystémov s prenosovými funkciami $G_1(s)$ a $G_2(s)$ je výstupom celkového systému jednoducho súčet výstupov podsystémov. Pre prenosovú funkciu celkového systému $G(s)$ platí
\begin{align}
    G(s) = G_1(s) + G_2(s)
\end{align}




\subsection{Spätnoväzbové zapojenie blokov}

Spätnoväzbové zapojenie blokov je znázornené na obr.~\ref{TFalgebra_spatnovazbove}. Pre lepšiu orientáciu je vstup celkového systému označený ako $u$ a výstup celkového systému ako $y$. Signál $y$ je vstupom spätnoväzbového podsystému $G_2(s)$. Takáto spätná väzba je odčítavaná (ide o zápornú spätnú väzbu) od vstupného signálu $u$. Vzniká odchýlkový signál $e$, ktorý je vstupom podsystému $G_1(s)$.


\begin{center}

    \makebox[\textwidth][c]{%
    \input{../fig_standalone/TFalgebra_spatnovazbove.pdf_tex}
    }

	\figcaption{Spätnoväzbové zapojenie blokov}
	\label{TFalgebra_spatnovazbove}

\end{center}

Bez uvádzania podrobností a predpokladov môžeme písať o odchýlkovom signále:
\begin{align}
    e = u - G_2(s) y
\end{align}
a potom
\begin{subequations}
    \begin{align}
        y &= G_1(s) e \\
        y &= G_1(s) \left( u - G_2(s) y \right) \\
        \left( 1 + G_1(s)G_2(s) \right) y &= G_1(s) u  \\
        y &= \frac{G_1(s)}{\left( 1 + G_1(s)G_2(s) \right)} u
    \end{align}
\end{subequations}
Pre prenosovú funkciu celkového systému $G(s)$ platí
\begin{align}
   G(s) = \frac{G_1(s)}{\left( 1 + G_1(s)G_2(s) \right)}
\end{align}












% \newpage


\section{Otázky a úlohy}

\begin{enumerate}[leftmargin=0pt, labelsep=3mm, itemsep=0pt]

    \item Napíšte vzťah (rovnicu), ktorým je definovaná Laplaceova transformácia.

    \item Napíšte Laplaceov obraz derivácie časovej funkcie $\frac{\text{d}f(t)}{dt}$.

    \item Napíšte Laplaceov obraz jednotkového skoku.

    \item Napíšte Laplaceov obraz Dirackovho impulzu.

    \item Nájdite analytické riešenie diferenciálnej rovnice s využitím Laplaceovej transformácie.
    \begin{align*}
        \dot y(t) + a_0 y(t) = b_0 u(t) \qquad y(0) = y_0 \qquad a_0, b_0, y_0\in\mathbb R \qquad u(t) = 1
    \end{align*}

    \item Nájdite analytické riešenie diferenciálnej rovnice s využitím Laplaceovej transformácie.
    \begin{align*}
        \dot y(t) + a_0 y(t) = b_0 u(t) \qquad y(0) = y_0 \qquad a_0, b_0, y_0\in\mathbb R \qquad u(t) = \delta(t)
    \end{align*}

    \item Nájdite analytické riešenie diferenciálnej rovnice s využitím Laplaceovej transformácie.
    \begin{align*}
        \ddot y(t) + (a+b) \dot y(t) + ab y(t) = 0 \qquad y(0) = y_0,\ \dot y(0) = z_0  \qquad a\in\mathbb R,\ b\in\mathbb R,\ y_0\in\mathbb R,\ z_0\in\mathbb R
    \end{align*}

    \item Nájdite analytické riešenie diferenciálnej rovnice s využitím Laplaceovej transformácie.
    \begin{align*}
        \ddot y(t) +4 \dot y(t) + 3y(t) = u(t) \qquad y(0) = 3,\ \dot y(0) = -2 \qquad u(t) = 1
    \end{align*}

    {\color{Gray} \scriptsize K dispozícii je tabuľka Laplaceových obrazov:

    \begin{center}

        \smallskip

        % \newcommand{\Laplace}[1]{\ensuremath{\mathcal{L}{\left\{#1\right\}}}}
        % \newcommand{\InvLap}[1]{\ensuremath{\mathcal{L}^{-1}{\left\{#1\right\}}}}
        \begin{tabular*}{0.68\textwidth}{c @{\extracolsep{\fill}} c}
            \toprule
            $f(t)$                                  & $\Laplace{f(t)}=F(s)$ \\
            \midrule
            $\displaystyle \frac{\text{d}^n f(t)}{\text{d}t^n}$ & $s^nF(s) - s^{(n-1)} f(0) - \cdots - f^{(n-1)}(0)$   \\ \addlinespace[2mm]
            $e^{at}$ 	                            & $\dfrac{1}{s-a}$     \\ \addlinespace[2mm]
            $1$                                     & $\dfrac{1}{s}$      \\ \addlinespace[2mm]
            $\delta(t)$	                            & $1$                 \\
            \bottomrule
        \end{tabular*}

    \end{center}
    }


\end{enumerate}











\printbibliography[title={Literatúra}]





\end{document}
